%----------------------------------------------------------------------------------------
%	PACKAGES AND THEMES
%----------------------------------------------------------------------------------------
\documentclass[aspectratio=169,xcolor=dvipsnames]{beamer}
\usetheme{SimplePlus}

\usepackage[utf8]{inputenc} 
\usepackage[T2A]{fontenc} 
\usepackage[russian]{babel}
\usepackage{amsmath,amssymb}
\usepackage{hyperref}
\usepackage{graphicx} % Allows including images
\usepackage{booktabs} % Allows the use of \toprule, \midrule and \bottomrule in tables
\usepackage{multicol}
\usepackage{expl3,xparse}
\usepackage{ebproof}
\usepackage{mathtools} 
\usepackage{extarrows} 
\usepackage{fitch}

%%%Настройка определений, теорем, лемм и пр
\usepackage{amsthm}
\theoremstyle{remark}
\theoremstyle{mystyle}
\newtheorem{mydef}{Определение}
\newtheorem{mylemma}[theorem]{Лемма}
\newtheorem{mycor}[theorem]{Утверждение}
%%%%%%%%%%%%%%%%%%%%%%%%%%%%%%%%%%%%%%%%%%%%%%%%%

\setbeamersize{text margin left=15pt,text margin right=15pt}

%Команды
%\newcommand{\mydef}[1]{\textbf{Определение. }\alert{#1}}
\newcommand{\mythm}[1]{\textbf{Теорема. }\alert{#1}}
\newcommand{\mylem}[1]{\textbf{Лемма. }\alert{#1}}
\newcommand{\myprop}[1]{\textbf{Утверждение. }\alert{#1}}

%Сокращения для названий языков
\newcommand{\EL}{\mathcal{EL}}
\newcommand{\ELC}{\mathcal{EL\text{-}C}}
\newcommand{\ELD}{\mathcal{EL\text{-}D}}
\newcommand{\ELCD}{\mathcal{EL\text{-}CD}}
\newcommand{\ELE}{\mathcal{EL\text{-}E}}
\newcommand{\PAL}{\mathcal{PAL}}
\newcommand{\PALC}{\mathcal{PAL\text{-}C}}
\newcommand{\PALD}{\mathcal{PAL\text{-}D}}
\newcommand{\PALCD}{\mathcal{PAL\text{-}CD}}
\newcommand{\ELRC}{\mathcal{EL\text{-}RC}}
\newcommand{\PALRC}{\mathcal{PAL\text{-}RC}}

%Еще сокращения
\newcommand{\metabot}{\text{<<}\bot\text{>>}}

%----------------------------------------------------------------------------------------
%	TITLE PAGE
%----------------------------------------------------------------------------------------
\subtitle{Мини-курс <<Эпистемическая логика: исчисления и модели>>}

\author[]{Виталий Долгоруков, Елена Попова}

\institute[] % Your institution as it will appear on the bottom of every slide, may be shorthand to save space
{Международная лаборатория логики, лингвистики и формальной философии НИУ ВШЭ}
\date{Летняя школа <<Логика и формальная философия>> \\ Факультет свободных искусств и наук \\ 
сентябрь 2022} % Date, can be changed to a custom date


%----------------------------------------------------------------------------------------
%	PRESENTATION SLIDES
%----------------------------------------------------------------------------------------

\subtitle{\textit{7. Эпистемическая логика (статическая)}}
%TODO картинка елочка для общего знания двух агентов

\begin{document}
\frame[plain]{\titlepage}
\frame{\frametitle{Оглавление}\tableofcontents}

\section{Основные идеи эпистемической логики}
\stitle{Основные идеи эпистемической логики}

\frame{\frametitle{Термины}
\pause
\begin{itemize}[<+->]
	\item Знание vs. мнение 
	\item Метазнание = знание о знании
	\item Эпистемическая логика = логика знания (и метазнания)
	\item Доксатическая логика = логика мнения (и метамнения)
	\end{itemize}
}

\frame{\frametitle {Пример : Информационное обновление} \pause
\begin{block}{Информационное обновление}
Мама Ангелина Петровна, папа Аристарх Платонович и малыш Петечка заказали три блюда (каждый – только одно):  \pause
 мороженое,  \pause
  рыбу фугу  \pause 
  и бифштекс с кровью. \pause
    \\
официант: \textit{рыба?}  \pause 
мама Ангелина Петровна: \textit{мне!}  \pause
 \\
официант: \textit{мороженое?}  \pause
 папа Аристарх Платонович: \textit{мне!}   \pause
  \\ 
официант: \textit{ага!  \pause 
значит, бифштекс с кровью малышу Петечке} 	
\end{block}
}

\frame{ \frametitle{Пример : Метазнание}
\begin{center}	
\includegraphics[width=0.9\linewidth]{images/Joe.jpg} \\
\href{http://youtu.be/YyfozcsTPmQ}{\underline{Что знает Джо?}}
\end{center}
}

\frame { \frametitle {Пример .  Информационное обновление + метазнание}

\begin{block}{Рассуждающие мудрецы}
\textit{На голову каждого из трех мудрецов надели шляпу (всего 5 шляп: 3 черных и 2 белых). Каждый мудрец может видеть шляпы на двух других мудрецах, но не знает какая шляпа на нем. Каждому мудрецу нужно догадаться, какая на нем шляпа и сказать ответ. Как же они это сделали?}
\end{block}
}

\frame { \frametitle {Логика: от теории дедукции к теории информационого обновления}
\vspace{-1cm}
\includegraphics[width=0.23\linewidth]{images/vanBenthem.jpg} \\
Й. ван Бентем: 
\begin{quotation}
<<Спрашивание и получение ответов являются такими же важными логическими формами деятельности, что и извлечение следствий!>>	
\end{quotation}
}

\frame { \frametitle {(Статическая) эпистемическая логика}  \pause

\begin{itemize}[<+->]
	\item агент $i$ знает, что  $\varphi$  \\
          $K_i \varphi$
    \item агент $i$ не  знает, что  $\varphi$  \\
            $\neg K_i \varphi$ 
    \item агент $i$ знает, что  $ \neg \varphi$  \\
           $K_i \neg \varphi$ 
    \item  агент $i$ не  знает, что  $ \neg \varphi$ \\
           агент $i$ допускает, что  $\varphi$  \\
           $ \neg K_i \neg \varphi$  \pause
            $= \hat{K}_{i} \varphi$      
\end{itemize}
}

\section{Синтаксис и семантика эпистемической логики}
\stitle{Синтаксис и семантика эпистемической логики}

\subsection{Синтаксис}
\frame { \frametitle{Правильно построенная формула}
\begin{block}{Язык эпистемической логики}
\centering
$ \varphi ::=   p \mid \neg \varphi \mid (\varphi \wedge \psi) \mid (\varphi \vee  \psi) \mid (\varphi  \to  \psi) \mid K_i \varphi  \mid \hat{K}_{i} \varphi $ 	
\end{block}
}

\subsection{Исчисление}
\frame{ \frametitle{Исчисление} \pause
\begin{block}{}
\begin{itemize}[<+->]
	\item[K] $K_i (\varphi \to \psi) \to (K_i \varphi \to K_i \psi)  $
	\item[T] $K_i \varphi \to \varphi $
	\item[4] $K_i \varphi \to K_i K_i \varphi $
	\item[5] $ \neg K_i \varphi  \to K_i \neg K_i \varphi $  
	\item[G] $ \underline{\vdash \varphi}$ \\
             $ \vdash K_i \varphi$  
\end{itemize}
\end{block}
}
%\subsubsection{Эпистемический шестиугольник}
%\frame{\frametitle{Эпистемический шестиугольник}
%\begin{center}
%\begin{tikzpicture}[scale=1.55, auto, ultra thick,main node/.style={fill=blue!20,rounded corners, font=\rmfamily\Large\bf}]
%\node[main node] (1) at (1,2) { $ \hat{K}_{i} \varphi$}; 
%\node[main node] (2) at (3,0.75) {$\hat{K}_{i} \varphi \wedge \hat{K}_{i}  \neg \varphi $};
%\node[main node] (3) at (5,2) {$\hat{K}_{i} \neg \varphi$}; 
%\node[main node] (4) at (5,4) {$ K_i \neg \varphi$}; 
%\node[main node] (5) at (3,5.25) {$ K_i \varphi \vee K_i \neg \varphi$}; 
%\node[main node] (6) at (1,4) {$ K_i \varphi $};
%\path
%(6) edge [-,HSEblue] node {} (2) 
%edge [-,HSEblue] node {} (4) 
%edge [->,black] node {} (1) 
%edge [->, black] node {} (5) 
%edge [-,red] node {} (3) 
%(1) edge [-,black!60!green] node {} (5) 
%edge [-,red] node {} (4) 
%edge [-,black!60!green] node {} (3) 
%(2) edge [->, black] node {} (1) 
%edge [-,HSEblue] node {} (4) 
%edge [->, black] node {} (3) 
%edge [-,red] node {} (5) 
%(3) edge [-,black!60!green] node {} (5) 
%(4) edge [->, black] node {} (3)
%(4) edge [->, black] node {} (5) ;
%\end{tikzpicture}
%\end{center}
%}

\subsection{Cемантика}
\frame {\frametitle{Модель}
\begin{block}{}
$\mathcal{M}=(\mathcal{A},W, \{ \sim_{i} \}_{i \in \mathcal{A}},V)$  \\  \pause 
\begin{itemize}
	\item  $\mathcal{A}$ – множество агентов 
	\item  $W$ – множество возможных миров (ситуаций)
	\item  $\sim_{i}$ – отношение на $W$ для агента  $i$ 
	\item  $V: Var \mapsto \mathcal{P}(W)$ 
\end{itemize}
\end{block}
}

\frame {\frametitle{Выполнимость в модели}
\begin{block}{}
 $\mathcal{M}, w \models p$ е.т.е. $ w \in V(p)$ \\ 
  $\mathcal{M}, w \models \varphi \wedge \psi $ е.т.е. $ \mathcal{M}, w \models \varphi$ и $ \mathcal{M}, w \models \psi$ \\ 
 $\mathcal{M}, w \models  \neg \varphi   $ е.т.е. $\mathcal{M}, w \not \models \varphi$ \\ 
для $\vee$ и $\to$ – аналогично \\ 
 $\mathcal{M}, w \models  K_i \varphi   $ е.т.е. $ \forall w' (w \sim_i w' \to \mathcal{M}, w'  \models \varphi) $ \\ 
 $\mathcal{M}, w \models  \hat{K}_{i} \varphi  $ е.т.е. $ \exists w' (w \sim_i w'  \wedge \mathcal{M}, w'  \models \varphi) $
 \end{block}
 }


\frame {\frametitle{Ограничения на $\sim_{i}$}
\begin{itemize}[<+->]
\item $K_i \varphi \to \varphi $ \\
\item $ \forall w' (w'  \sim_{i} w') $ \\ 
\item $K_i\varphi  \to K_i K_i \varphi $ \\
\item $ \forall w' \forall w''\forall w''' ( (w' \sim_{i} w'' \wedge  w'' \sim_{i} w''') \to w' \sim_{i} w''')$ \\
\item $ \neg K_i \varphi  \to K_i \neg K_i \varphi $ \\
\item $ \forall w' \forall w'' ( (w' \sim_{i} w'' \wedge w'\sim_{i} w''') \to  w'' \sim_{i} w''')$ 

\end{itemize}
}

\frame{\frametitle {$a$ знает, что $p$}
\centering
\begin{multicols}{2}
 $\mathcal{M}_1, w_1 \models K_a p$ 
\columnbreak   
\begin{tikzpicture}[modal, node distance=3.7cm]
\node[world](1)              {$\underline{w_1}: p$};  \pause
\node[world](2)[right of=1]  {$w_2: p$};  \pause
\path[<->] (1) edge[loop above] node{$a$}(1);  \pause
\path[<->] (2) edge[loop above] node{$a$}(2);  \pause
\path[<->] (1) edge             node{$a$}(2); 
\end{tikzpicture}
\end{multicols}
}


\frame { \frametitle {$a$ не знает, что  $p$}
\pause $\mathcal{M}_2, w_1 \models \neg K_a p$ \\  \pause
\begin{center}
\begin{tikzpicture}[modal, node distance=3.7cm]
\node[world](1)               [label= below left:{$\underline{w_1}$}]{$p$};  \pause
\node[world](2) [right of=1]  [label=below left:{$w_2$}]{$\neg p$};  \pause

\path[<->] (1) edge[loop above] node{$a$}(1);  \pause
\path[<->] (2) edge[loop above] node{$a$}(2);  \pause
\path[<->] (1) edge             node{$a$}(2); 
\end{tikzpicture}
\end{center}
}

\sstitle{Несколько примеров}

\frame{\frametitle{Пример : <<Пьеро и Мальвина>>}

<<Пьеро любит Мальвину ($l$), но она об этом не знает>> \pause 
\begin{center}
$l \wedge K_p l \wedge \neg K_m l \wedge \neg K_m \neg l$	
\end{center}
\pause 
\begin{center}
\begin{tikzpicture}[modal, node distance=3.7cm, font=\small]
\pause
\node[world] (1) [label= below left:{$\underline{w_1}$}]{$l$};
\pause
\node[world] (2) [right of=1] [label= below left:{$w_2$}]{$\neg l$};
\pause
\path[<->](1) edge [HSEblue]          node {m} (2);  
\pause
\path[<->](1) edge [HSEblue, loop above]  node {m} (1); 
\pause
\path[<->](2) edge [HSEblue, loop above]  node {m} (2); 
\pause
\path[<->](1) edge [red, loop left]  node {p} (1);
\pause
\path[<->](2) edge [red, loop right]  node {p} (2); 
\end{tikzpicture}
\end{center}
}

\frame{\frametitle{Пример : <<Пьеро и Мальвина>>}

<<Пьеро любит Мальвину ($l$), но она об этом не знает>>
\begin{center}
$l \wedge K_p l \wedge \neg K_m l \wedge \neg K_m \neg l$	
\end{center}


\begin{center}
\begin{tikzpicture}[modal, node distance=3.7cm, font=\small]
\node[world] (1) [label= below left:{$\underline{w_1}$}]{$l$};
\node[world] (2) [right of=1] [label= below left:{$w_2$}]{$\neg l$};
\path[<->](1) edge [HSEblue]          node {m} (2);  
\end{tikzpicture}
\end{center}
\footnotesize{Петли подразумеваются во всех мирах для всех агентов, их можно не изображать.}
}

\frame{\frametitle{Пример 5: <<прагматика вопроса>>}
$a$ спрашивает у $b$: <<это здание Вышки?>> 	
\pause 
\begin{center}
\begin{tikzpicture}[modal, node distance=3.7cm, font=\small]
\node[world] (1) [label= below left:{$\underline{w_1}$}]{$p$};
\node[world] (2) [right of=1] [label= below left:{$w_2$}]{$\neg p$};
\path[<->](1) edge [HSEblue]          node {a} (2);  
\end{tikzpicture}
\end{center}
}

\frame[plain]{\frametitle{Пример : <<сложные отношения>>}
\pause 
<<Аня любит Борю, но он об этом не знает, а Боря любит Аню, но она об этом не знает>>	
\pause  
\begin{center}
\begin{tikzpicture}[modal, node distance=3.5cm]

\node[world] (1) [label= below left:{$\underline{w_1}$}]{$p, q$};
\node[world] (2) [right of=1] [label= below left:{$w_2$}]{$p, \neg q$};
\node[world] (3) [below  of=1][label= below left:{$w_3$}]{$\neg p, q$};
\node[world] (4) [right of=3][label= below left:{$w_4$}] {$\neg p, \neg q$};

\path[<->](1) edge [HSEblue]             node       {a} (2);
\path[<->](3) edge [HSEblue]             node[below]{a} (4);
\path[<->](1) edge [red]              node[left] {b} (3);
\path[<->](2) edge [red]              node       {b} (4);

\end{tikzpicture}
\end{center} }

\frame { \frametitle{Формулы и модели}
\begin{block}{Упражнение}
Построить модель для следующих формул (или доказать, что это невозможно)\\ \pause
\begin{multicols}{2}
\begin{enumerate}[<+->]
	\item $\neg K_a \varphi  \wedge K_b \varphi $
	\item   $\neg K_a \varphi  \wedge \varphi $ 
	\item $ K_a \varphi  \wedge \neg \varphi $
	\item  $ \hat{K}_a \varphi  \wedge \neg \varphi $
	\item $K_a \varphi  \wedge  ( \varphi \to \psi)  \wedge \neg K_a \psi $
	\item $K_a \varphi  \wedge  K_a ( \varphi \to \psi)  \wedge \neg K_a\psi $ 
	\item $ \hat{K}_a \varphi  \wedge \hat{K}_a \neg \varphi  $
	\item $ K_a K_b \varphi \wedge \neg K_a \varphi$
	\item $ K_a K_b \varphi  \wedge \neg  K_b K_a \varphi  $
	\item $ K_b K_a \varphi  \wedge K_a K_b \varphi \wedge   \neg K_a K_b  K_a \varphi  $ 
\end{enumerate}
\end{multicols}
\end{block}

}


\section{Формы группового (коллективного) знания}
\stitle{Формы группового (коллективного) знания}

\frame{ \frametitle{Групповое знание}

\begin{block}{Итерация знания} \pause
$K_{i}^{n}\varphi  := \underbrace{K_i ... K_i}_{n раз}\varphi$  	
\end{block}

\pause

\begin{block}{Пример} \pause
$K_{a}^{3} \varphi=K_a K_a K_a \varphi$ \\ \pause
Для любого $i$ верно, что $K_{i}^{n} \varphi=K_i \varphi$	
\end{block}
}

\subsection{<<Все знают>> (распространенное знание)}

\frame{ \frametitle{<<Все знают>> (распространенное знание)}
\begin{block}{Распространенное знание:}
\begin{center}
$E_{G} \varphi := \bigwedge \limits_{i \in G} K_i \varphi$			
\end{center}
\end{block}
\pause
\begin{block}{Пример} \pause
$E_{ab} \varphi = K_a \varphi \wedge K_b \varphi$ 	
\end{block}
}

\frame{\frametitle{<<Все знают>>}
\begin{block}{Что значит $E_{ab}^{2} \varphi$?} \pause
$E_{ab}^{2} \varphi = E_{ab} E_{ab} \varphi$  \pause
 $= E_{ab} (K_a \varphi \wedge K_b \varphi)$  \pause 
 $= K_a K_b \varphi \wedge K_b K_a \varphi \wedge K_a K_a \varphi \wedge K_b K_b \varphi$  \pause
  $= K_a \varphi \wedge K_b \varphi \wedge K_a K_b \varphi \wedge K_b K_a \varphi $ \\  \pause	
\end{block}
\pause 
\begin{block}{Упражнение:}
Построить модель для формулы: $E_{ab} \wedge \neg E_{ab}^{2}$	
\end{block}
}

\subsection{Общее знание}
\frame{\frametitle{<<Общее знание>>}
\begin{block}{<<Общее знание>> (<<common knowledge>>)}
$C_G \varphi := \bigwedge \limits_{i=0}^{\infty} E_{G}^{i}  \varphi = \varphi \wedge  E_{G} \varphi \wedge E_{G}^{2} \varphi \wedge E_{G}^{3} \varphi ...  $	
\end{block}
}

%TODO Общее знание: сделать лесенку
\frame{\frametitle{Пример : <<Дед Мороз>>} 
\begin{block}{Ситуация:} \textit{Пятилетний Андрюша уже знает, что Деда Мороза не существует \pause и он знает, что папа (Борис) знает, что Деда Мороза не существует. \pause Папа знает, что Андрюша знает правду о Деде Морозе. \pause  Но Андрюша пока не знает, знает ли папа, что он уже знает правду. }
\end{block}
\pause 
\begin{block}{Формализация:} \pause
\begin{multicols}{3}
\begin{enumerate}[<+->]
    \item $K_a p$
    \item $K_b  p$ 
    \item $E_{ab} p$
    \item $K_a K_b p$
    \item $K_b K_a p$
    \item $E^2_{ab} p$
    \item $\neg K_a K_b K_a p$ 
    \item $\neg E^3_{ab} p$
	\item $ \neg C_{ab} p$
\end{enumerate}
\end{multicols}
\end{block}
}



%TODO Общее знание: Льюис, Грайс, экономисты (лекция Соколова), Томаселло


\frame{\frametitle{Пример : <<Сплетни>>}
\begin{block}{Ситуация:} \textit{Аня рассказала Боре страшный секрет ($p$). Боря рассказал его Свете, но попросил Свету не рассказывать об этом Ане. Света ничего не сказала Ане.}
\end{block}
\pause 
\begin{block}{Формализация:}
\begin{multicols}{3}
\begin{enumerate}[<+->]
\item $K_a p$
\item $K_b p$
\item $K_c p$
\item $E_{abc} p$
\item $K_a K_b p$
\item $K_b K_a p$
\item $K_a K_b K_a p$
\item $K_b K_a K_b p$
\item $E^{10}_{ab} p$
\item $C_{ab} p$
\item $K_b K_c K_b K_c p$
\item $E^{100}_{bc} p$
\item $C_{bc} p$
\item $\neg K_a K_c p$
\item $\neg E^2_{ac}p$
\item $\neg C_{ac} p$
\item $\neg C_{abc} p$
\end{enumerate}
\end{multicols}	
\end{block}
} 

\subsection{Дистрибутивное знание}
\frame{\frametitle{<<Дистрибутивное (распределенное) знание>> }
\begin{block}{<<Дистрибутивное (распределенное) знание>>:}
знание, которое есть у группы потенциально, если бы все агенты обменялись информацией \\ \pause
%$D_G \varphi $ е.т.е. существует такое $G' \subseteq G$, что  для каждого $i \in G'$ найдется формула $\varphi_i$ такая, что 1) $K_i \varphi_i$ и 2) $ (\bigwedge \varphi_i \to \varphi)$
\end{block}
\begin{block}{Пример}\pause
$K_a \varphi \wedge K_b (\varphi \to \psi)$ \\  
$D_{ab} \psi$		
\end{block}

}

%%%%%%%%%%%%%%%%%%%%%%%%%%%%%%%%%%%%%%%%%%%%%%%%%%%%%%%%
\frame{\frametitle{Определения форм группового знания}
\begin{block}{Групповое отношение достижимости} \pause 
\begin{itemize}[<+->]
\item $R_{E_{G}}:=\bigcup \limits_{ i \in G} \sim_i$ 
\item $R_{D_{G}}:=\bigcap \limits_{ i \in G} \sim_i$ 
\item  $R_{C_{G}}: = (\bigcup \limits_{ i \in G} \sim_i)^{*}$, т.е. рефлексивное транзитивное замыкание отношения $R_{E_{G}}$ \\ \pause
\footnotesize{(Пример транзитивного замыкания: Пусть $R$ – отношение <<быть родителем>>, тогда транзитивное замыкание этого отношения – отношение <<быть предком>>)}.
\end{itemize}
\end{block}
}

\frame{\frametitle{*Определения форм группового знания}
\begin{block}{Три вида группового знания:} \pause
\begin{itemize}[<+->]
\item $M, w_i \models E_G \varphi $ е.т.е. $\forall w' (w_i R_{E_{G}} w' \to M, w' \models \varphi)$ 
\item $M, w_i \models D_G \varphi $ е.т.е. $\forall w' (w_i R_{D_{G}} w' \to M, w' \models \varphi)$ 
\item $M, w_i \models C_G \varphi $ е.т.е. $\forall w' (w_i R_{C_{G}} w' \to M, w' \models \varphi)$ 	
\end{itemize}
\end{block}
}

%%%%%%%%%%%%%%%%%%%%%%%%%%%%%%%%%%%%%%%%%%%%%%%%%%%%%%%%%%%
\frame{\frametitle{*Некоторые свойства}
\begin{block}{Аксиомы для общего знания:}\pause
\begin{itemize}[<+->]
  \item $C_{G} (\varphi \to \psi) \to (C_{G}  \varphi \to C_{G} \psi)$
  \item $C_{G} \varphi \to (\varphi \wedge E_{G}C_{G}\varphi)$
  \item $C_{G}(\varphi \to E_{G} \varphi) \to (\varphi \to C_{G} \varphi ) $
\end{itemize}	
\end{block}
}

\frame{\frametitle{Формулы и модели}
\begin{block}{Упражнение:}
Построить модель для следующих формул:
\begin{enumerate}
\item $E_{abc} \varphi \wedge \neg K_aK_b \varphi$
\item $E_{ab} \varphi \wedge \neg E_{ab} E_{ab} \varphi $ 
\item $D_{ab} \varphi \wedge \neg K_a \varphi  \wedge \neg K_b \varphi $ 
\item $C_{ab} \varphi \wedge \neg K_c \varphi$ 
\item $E^{2}_{ab}\varphi \wedge \neg C_{ab} \varphi$
\end{enumerate}
\end{block}
}

\frame{\frametitle{Что откуда следует?}
У нас есть следующие операторы:
\begin{center}
$D_G$, $K_i$, $C_G$, $E_G$, $E^n_G$	
\end{center}
Как они связаны между собой?  \\  \pause 
\begin{block}{Иерархия форм группового знания ($i \in G$):}
\begin{center}
$C_G \varphi \to E^n_G \varphi \to E^{n-1}_G \varphi \to \dots \to E_G \varphi \to K_i \varphi \to D_G \varphi \to \varphi $		
\end{center}
\end{block}
}

\section{Дополнительно}
\frame{\frametitle{Дополнительно}
\begin{itemize}
  \item \href{https://postnauka.ru/video/103700}{Общее знание (в социальных науках) Михаил Соколов / ПостНаука}
  \item \href{https://www.youtube.com/watch?time_continue=274&v=eay1-m7RpoU}{Steven Pinker: The Elephant, the Emperor, and the Matzo Ball (про общее знание в психологии)}
  \item \href{https://plato.stanford.edu/entries/logic-epistemic/}{SEP: Epistemic Logic}
  \item \href{https://plato.stanford.edu/entries/common-knowledge/}{SEP: Common Knowledge}
  \item \href{https://www.youtube.com/watch?v=gLqBmpKCncw}{Mutual Knowledge}
  \item \href{https://philpapers.org/browse/epistemic-logic/}{Philpapers: Epistemic Logic}
  \item \href{https://plato.stanford.edu/entries/formal-epistemology/}{SEP: Formal Epistemology}
  \item \href{https://philpapers.org/browse/formal-epistemology/}{Philpapers: Formal Epistemology}
  \item \href{https://plato.stanford.edu/entries/epistemic-paradoxes/}{SEP: Epistemic Paradoxes}
  \item \href{https://plato.stanford.edu/entries/fitch-paradox/}{SEP: Fitch’s Paradox of Knowability}
  \item \href{https://www.youtube.com/watch?v=OM5awmxJh-s}{Vincent Hendricks: The trouble with bubbles}
  \end{itemize}
}

%\frame{\frametitle{Оглавление} \tableofcontents}

\end{document}