%----------------------------------------------------------------------------------------
%	PACKAGES AND THEMES
%----------------------------------------------------------------------------------------
\documentclass[aspectratio=169,xcolor=dvipsnames]{beamer}
\usetheme{SimplePlus}

\usepackage[utf8]{inputenc} 
\usepackage[T2A]{fontenc} 
\usepackage[russian]{babel}
\usepackage{amsmath,amssymb}
\usepackage{hyperref}
\usepackage{graphicx} % Allows including images
\usepackage{booktabs} % Allows the use of \toprule, \midrule and \bottomrule in tables
\usepackage{multicol}
\usepackage{expl3,xparse}
\usepackage{ebproof}
\usepackage{mathtools} 
\usepackage{extarrows} 
\usepackage{fitch}

%%%Настройка определений, теорем, лемм и пр
\usepackage{amsthm}
\theoremstyle{remark}
\theoremstyle{mystyle}
\newtheorem{mydef}{Определение}
\newtheorem{mylemma}[theorem]{Лемма}
\newtheorem{mycor}[theorem]{Утверждение}
%%%%%%%%%%%%%%%%%%%%%%%%%%%%%%%%%%%%%%%%%%%%%%%%%

\setbeamersize{text margin left=15pt,text margin right=15pt}

%Команды
%\newcommand{\mydef}[1]{\textbf{Определение. }\alert{#1}}
\newcommand{\mythm}[1]{\textbf{Теорема. }\alert{#1}}
\newcommand{\mylem}[1]{\textbf{Лемма. }\alert{#1}}
\newcommand{\myprop}[1]{\textbf{Утверждение. }\alert{#1}}

%Сокращения для названий языков
\newcommand{\EL}{\mathcal{EL}}
\newcommand{\ELC}{\mathcal{EL\text{-}C}}
\newcommand{\ELD}{\mathcal{EL\text{-}D}}
\newcommand{\ELCD}{\mathcal{EL\text{-}CD}}
\newcommand{\ELE}{\mathcal{EL\text{-}E}}
\newcommand{\PAL}{\mathcal{PAL}}
\newcommand{\PALC}{\mathcal{PAL\text{-}C}}
\newcommand{\PALD}{\mathcal{PAL\text{-}D}}
\newcommand{\PALCD}{\mathcal{PAL\text{-}CD}}
\newcommand{\ELRC}{\mathcal{EL\text{-}RC}}
\newcommand{\PALRC}{\mathcal{PAL\text{-}RC}}

%Еще сокращения
\newcommand{\metabot}{\text{<<}\bot\text{>>}}

%----------------------------------------------------------------------------------------
%	TITLE PAGE
%----------------------------------------------------------------------------------------
\subtitle{Мини-курс <<Эпистемическая логика: исчисления и модели>>}

\author[]{Виталий Долгоруков, Елена Попова}

\institute[] % Your institution as it will appear on the bottom of every slide, may be shorthand to save space
{Международная лаборатория логики, лингвистики и формальной философии НИУ ВШЭ}
\date{Летняя школа <<Логика и формальная философия>> \\ Факультет свободных искусств и наук \\ 
сентябрь 2022} % Date, can be changed to a custom date


%----------------------------------------------------------------------------------------
%	PRESENTATION SLIDES
%----------------------------------------------------------------------------------------


\title{Полнота $PAL$}

\begin{document}  \frame{\titlepage}
\section{$PAL$}

\frame{\frametitle{Язык}

\begin{block}{Язык $\PAL$}
\centering
$\varphi, \psi::= p \mid \neg \varphi  \mid (\varphi \wedge \psi) \mid K_i \varphi \mid [!\varphi]\psi$	
\end{block}
}

\frame{\frametitle{Семантика языка $\PAL$}

\begin{block}{Определение}
$M, w \models [! \varphi]\psi$  е.т.е.  $M, w \models \varphi  \Rightarrow M^{!\varphi}, w \models \psi$	
\end{block}

\begin{block}{Определение}
Если $M =(W, \{ \sim_i \}_{i\in Ag}, V)$ – модель эпистемической логики, то  $M^{!\varphi} =(W^{!\varphi},\{ \sim^{!\varphi}_i \}_{i\in Ag}, V^{!\varphi})$ – обновленная модель относительно $!\varphi$, где  	
\begin{itemize}
\item $W^{!\varphi} = \{w \in W \mid M, w \models \varphi \}$
\item $\sim^{!\varphi}_i = \; \sim_i \cap \; (W^{!\varphi} \times W^{!\varphi}) $
\item $V^{!\varphi}(p) = V(p) \cap \; W^{!\varphi} $
\end{itemize}

\end{block}

}


%\frame{\frametitle{Некоторые законы $\PAL$}
%}

\frame{\frametitle{Исчисление $PAL$ ($S5_m[]$)}
\begin{block}{}
Аксиомные схемы:
\begin{itemize}
\item Аксиомные схемы $S5_m$
\item $[!\varphi]p \leftrightarrow (\varphi \to p)$	
\item $[!\varphi]\neg \psi  \leftrightarrow (\varphi \to \neg [!\varphi]\psi)$
\item $[!\varphi](\psi \wedge \chi) \leftrightarrow ([!\varphi]\psi \wedge [!\varphi]\chi) $
\item $[!\varphi]K_i \psi \leftrightarrow (\varphi \to K_i [!\varphi] \psi)$
\end{itemize}
Правила вывода:
$MP$, $NEC$, $RE!$
\end{block}

}


\frame{\frametitle{Исчисление $PAL'$ ($S5_m[]'$)}
\begin{block}{}
Аксиомные схемы:
\begin{itemize}
\item Аксиомные схемы $S5_m$
\item $[!\varphi]p \leftrightarrow (\varphi \to p)$	
\item $[!\varphi]\neg \psi  \leftrightarrow (\varphi \to \neg [!\varphi]\psi)$
\item $[!\varphi](\psi \wedge \chi) \leftrightarrow ([!\varphi]\psi \wedge [!\varphi]\chi) $
\item $[!\varphi]K_i \psi \leftrightarrow (\varphi \to K_i [!\varphi] \psi)$
\item $[!\varphi][!\psi]\chi \leftrightarrow [!(\varphi \wedge [!\varphi]\psi )]\chi$
\end{itemize}
Правила вывода:
$MP$, $NEC$
\end{block}

}

\frame{\frametitle{Корректность $PAL$}
Упражнение
%$\models [!\varphi]p \leftrightarrow (\varphi \to p)$
%\begin{enumerate}
%\item $M, x \models [!\varphi]p $
%\item $M, x \models \varphi \Rightarrow M^{!\varphi}, x \models p$	
%\end{enumerate}


}

\frame{\frametitle{Выразительность}
\begin{description}
\item[$\EL$] $\varphi, \psi::= p \mid \neg \varphi  \mid (\varphi \wedge \psi) \mid K_i \varphi$
\item[$\PAL$] $\varphi, \psi::= p \mid \neg \varphi  \mid (\varphi \wedge \psi) \mid K_i \varphi \mid [!\varphi]\psi$
\end{description}
\centering
\begin{block}{Утверждение} $\EL$ и $\PAL$ совпадают по выразительность ($\EL \equiv \PAL $).
\end{block}

}

\frame{\frametitle{Перевод $tr$}
\begin{description}
\item[($\EL$)] $\varphi, \psi::= p \mid \neg \varphi  \mid (\varphi \wedge \psi) \mid K_i \varphi$
\item[($\PAL$)] $\varphi, \psi::= p \mid \neg \varphi  \mid (\varphi \wedge \psi) \mid K_i \varphi \mid [!\varphi]\psi$
\end{description}

\begin{mydef}[Функция перевода $tr: \PAL \mapsto \EL $]
\begin{multicols}{2}
\begin{itemize}
\item $tr(p):= p$	
\item $tr(\neg \varphi):= \neg tr(\varphi) $
\item $tr(\varphi \wedge \psi):= tr (\varphi) \wedge tr(\psi) $
\item $tr(K_i \varphi):= K_i tr(\varphi)$
\item $tr([!\varphi]p):= tr(\varphi \to p)$
\item $tr([!\varphi]\neg \psi):= tr(\varphi \to \neg [!\varphi]\psi)$
\item $tr([!\varphi](\psi \wedge \chi)):= tr([!\varphi]\psi \wedge [!\varphi]\chi) $
\item $tr([!\varphi]K_i \psi):= tr(\varphi \to K_i [!\varphi] \psi)$
\item $tr([!\varphi][!\psi]\chi):= tr([!\varphi]tr([!\psi]\chi))$
\end{itemize}
\end{multicols}
\end{mydef}

\begin{exampleblock}{Упражнение}
\begin{itemize}
\item $tr(\varphi \to \psi) = \dots$
\item  $tr(\varphi \vee  \psi) = \dots$		
\end{itemize}

\end{exampleblock}
}



\frame{\frametitle{Перевод}

\begin{exampleblock}{Примеры}
\begin{itemize}
\item $tr([!p](q \wedge r)) = tr([!p]q \wedge [!p]r) = tr([!p]q) \wedge tr([!p]r) = (p \to q) \wedge (p \to r) $
\item $tr([!p][!q]r)=tr([!p]tr([!q]r)) = tr([!p](q \to r)) = p \to (q \to r)$
\item $tr([!p]K_a q) = tr (p \to K_a[!p]q) = 
tr(p) \to  tr (K_a[!p]q) = p \to K_a tr ([!p]q) = 
p \to K_a (p \to q)$	
\end{itemize}

\end{exampleblock}
}

\frame{\frametitle{Сложность формулы $c$}

$\varphi \to K_i [!\varphi] \psi $ vs. $[!\varphi] K_i  \psi $

\begin{mydef}[Сложность формулы] Определим функцию $c: \PAL \mapsto \mathbb{N}$:
\begin{enumerate}
	\item $c(p):= 1$
	\item $c(\neg \varphi):= c(\varphi) + 1$
	\item $c(\varphi \wedge \psi) = max\{c(\varphi),c(\psi)\} + 1$
	\item $c(K_i \varphi):= c(\varphi) + 1$
	\item $c([!\varphi]\psi):= (c(\varphi) + 4) \cdot c(\psi)$
\end{enumerate}	
\end{mydef}
}


\frame{
\begin{block}{Лемма об уменьшении сложности формулы}
\begin{itemize}
\item $c(\varphi) \geq c(\psi)$ для $\psi \in Sub(\varphi)$
\item $c([!\varphi]p) > c(\varphi \to p)$ 
\item $c([!\varphi]\neg \psi) > c(\varphi \to \neg [!\varphi] \psi)$
\item $c([!\varphi][!\psi]\chi) > c([!\varphi]tr([!\psi]\chi))$ 
\end{itemize}
\end{block}
}

\frame{
\begin{block}{Теорема}
$ \forall \varphi \in \PAL \; \vdash_{PAL} \varphi' \leftrightarrow tr(\varphi')$	
\end{block}

\alert{$\blacktriangleright$} Рассмотрим следующие случаи: $\varphi' = p, \neg \varphi, \varphi \wedge \psi,   K_i \varphi, [!\varphi]p, [!\varphi]\neg \psi, [!\varphi](\psi \wedge \chi),  [!\varphi]K_i \psi,  [!\varphi][!\psi]\chi$
\begin{itemize}[<+->]
\item Случай $\varphi'= \alert{p}$. По определению $p \leftrightarrow tr(p)$. 
\item Случай $\varphi'= \alert{\neg \varphi}$. $\varphi \leftrightarrow tr(\varphi)$ по IH, поскольку $c(\varphi)<c(\neg \varphi)$
$$\neg \varphi \xleftrightarrow{} \neg tr(\varphi) \xleftrightarrow{}tr(\neg \varphi)$$
\item Случай $\varphi'= \alert{\varphi \wedge \psi}$. 
$\varphi \leftrightarrow tr(\varphi)$, $\psi \leftrightarrow tr(\psi)$
$$(\varphi \wedge \psi) \leftrightarrow
(tr(\varphi) \wedge tr(\psi)) \leftrightarrow
tr(\varphi \wedge \psi)$$

\item Случай $\varphi'= \alert{K_i \varphi}$. $\varphi \leftrightarrow tr(\varphi)$
$$K_i \varphi \leftrightarrow K_i tr(\varphi) \leftrightarrow tr (K_i \varphi)$$
\end{itemize}
}

\frame{
\begin{itemize}
\item Случай $\varphi'= \alert{[!\varphi]p}$.
\item Случай $\varphi'= \alert{[!\varphi]\neg \psi}$.
\item Случай $\varphi'= \alert{[!\varphi](\psi \wedge \chi)}$.	
\end{itemize}
}


\frame{
\begin{itemize}
\item Случай $\varphi'= \alert{[!\varphi]K_i \psi}$. 
$c(\varphi \to K_i [!\varphi]\psi) < c([!\varphi]K_i \psi)$
\begin{center}
$\vdash_{PAL} [!\varphi]K_i \psi \xleftrightarrow{\text{акс.}} 
(\varphi \to K_i [!\varphi]\psi) \xleftrightarrow{IH}
tr(\varphi \to K_i [!\varphi]\psi)) \xleftrightarrow{def}
tr([!\varphi]K_i \psi)$ 
\end{center}
\item Случай $\varphi'= \alert{[!\varphi][!\psi]\chi}$

\begin{enumerate}
\item $c([!\psi]\chi) < c([!\varphi][!\psi]\chi) $
\item $c([!\varphi]tr([!\psi]\chi)) < c([!\varphi][!\psi]\chi) $ 
\item $\vdash_{PAL} [!\psi]\chi \leftrightarrow tr([!\psi]\chi)$ IH 	из 1  
\end{enumerate}
\begin{center}
$\vdash_{PAL}[!\varphi][!\psi]\chi \xleftrightarrow[3]{!RE} [!\varphi]tr([!\psi]\chi) \xleftrightarrow[2]{IH} tr([!\varphi]tr([!\psi]\chi)) \xleftrightarrow[]{def} tr ([!\varphi][!\psi]\chi) $ 
\end{center}
\end{itemize}
\alert{$\blacktriangleleft$}
}

\frame{\frametitle{Сборка доказательства}

\begin{block}{Теорема о полноте $PAL$}
$\forall \varphi \in \PAL$:
\begin{center}
$\models_{C_{S5}} \varphi    \Rightarrow \;
\models_{C_{S5}} tr(\varphi) \Rightarrow \;
\vdash_{S5_m} tr(\varphi)  \Rightarrow \;
\vdash_{PAL} tr(\varphi) \Rightarrow \;
\vdash_{PAL} \varphi   
$	
\end{center}
\end{block}
}

%\frame{\frametitle{Лаконичность}
%\begin{block}{}
%$K_i \top$
%\end{block}
%}
%
%\frame{\frametitle{Другие варианты аксиоматизации $PAL$}
%$PAL_1$, $PAL_2$,$PAL_3$, 
%Подробнее Wang
%}

\frame{\frametitle{Дедуктивная эквивалентность}

\myprop{$\vdash_{PAL} \varphi  \Leftrightarrow \; \vdash_{PAL'} \varphi$}
В силу теоремы о полноте.




\begin{block}{Вопрос: Как доказать дедуктивную эквивалентность $PAL$ и $PAL'$ непосредственно?}
Задача сводится к том, чтобы вывести $COM!$ в $PAL_1$ и $RE!$ в $PAL_2$. Как это сделать? Гипотеза: индукцией по $c(\varphi)$.
\end{block}

}

\end{document}