%----------------------------------------------------------------------------------------
%	PACKAGES AND THEMES
%----------------------------------------------------------------------------------------
\documentclass[aspectratio=169,xcolor=dvipsnames]{beamer}
\usetheme{SimplePlus}

\usepackage[utf8]{inputenc} 
\usepackage[T2A]{fontenc} 
\usepackage[russian]{babel}
\usepackage{amsmath,amssymb}
\usepackage{hyperref}
\usepackage{graphicx} % Allows including images
\usepackage{booktabs} % Allows the use of \toprule, \midrule and \bottomrule in tables
\usepackage{multicol}
\usepackage{expl3,xparse}
\usepackage{ebproof}
\usepackage{mathtools} 
\usepackage{extarrows} 
\usepackage{fitch}
\theoremstyle{plain}
\newtheorem{mydef}{Определение}
\setbeamersize{text margin left=15pt,text margin right=15pt}

%Сокращения для названий языков
\newcommand{\EL}{\mathcal{EL}}
\newcommand{\ELC}{\mathcal{EL\text{-}C}}
\newcommand{\ELD}{\mathcal{EL\text{-}D}}
\newcommand{\ELCD}{\mathcal{EL\text{-}CD}}
\newcommand{\ELE}{\mathcal{EL\text{-}E}}
\newcommand{\PAL}{\mathcal{PAL}}
\newcommand{\PALC}{\mathcal{PAL\text{-}C}}
\newcommand{\PALD}{\mathcal{PAL\text{-}D}}
\newcommand{\PALCD}{\mathcal{PAL\text{-}CD}}
\newcommand{\ELRC}{\mathcal{EL\text{-}RC}}
\newcommand{\PALRC}{\mathcal{PAL\text{-}RC}}

%----------------------------------------------------------------------------------------
%	TITLE PAGE
%----------------------------------------------------------------------------------------
\subtitle{Мини-курс <<Эпистемическая логика: исчисления и модели>>}

\author[]{Виталий Долгоруков, Елена Попова}

\institute[] % Your institution as it will appear on the bottom of every slide, may be shorthand to save space
{Международная лаборатория логики, лингвистики и формальной философии НИУ ВШЭ}
\date{Летняя школа <<Логика и формальная философия>> \\ Факультет свободных искусств и наук \\ 
сентябрь 2022} % Date, can be changed to a custom date


%----------------------------------------------------------------------------------------
%	PRESENTATION SLIDES
%----------------------------------------------------------------------------------------
 
\usepackage{expl3, xparse}
\usepackage{ebproof}

\title{Эпистемическая логика с дистрибутивным знанием}

\begin{document}  \frame{\titlepage}
\section{Исчисление $K_{m}D$}

\frame{\frametitle{Исчисление $K_{m}D$}
\begin{enumerate}
\item Аксиомные схемы $K$ для $K_i$
\item Аксиомные схемы $K$ для $D_G$
\item $K_i \varphi \leftrightarrow D_{i} \varphi$
\item $D_G \varphi \to D_{G'} \varphi$	, где $G \subseteq G'$
\item Правило Гёделя для $G$
\end{enumerate}
}

\frame{\frametitle{Каноническая модель: первое приближение}

Каноническая модель

\begin{block}{}Каноническая достижимость: $X R^c_G Y:= \forall \varphi \in \ELD: D_G \varphi \in X \Rightarrow \varphi \in Y$
\end{block}

Но! Не гарантируется, что $R^c_G = \bigcap \limits_{i \in G} R^c_i$  
}

\frame{\frametitle{Распутывание модели в дерево (tree unraveling)}
Пример с одной модальностью.
Пути:  $\langle x \rangle$, $\langle y \rangle$, $\langle z \rangle$, 
$\langle w \rangle$, $\langle x, a, y \rangle$, $\langle x, a, y, b, w \rangle $ и т.д. 
}

\frame{\frametitle{Каноническая псевдомодель $\to$ каноническая модель}
Мы будем понимать $K_i$ как синоним для $D_{i}$
\begin{block}{}\alert{Определение}.
\textbf{Каноническая псевдомодель} $M^c = (W^c, R^c_{G_1}, \dots,  R^c_{G_n},  V^c)$, где 
\begin{itemize}
\item $\emptyset \not = G_i \subseteq Ag $ 
\item $W^c = \{ X \mid X - \text{м.н.м.} \}$
\item $XR^c_GY \Leftrightarrow \forall \varphi (D_G \varphi \in X \Rightarrow \varphi \in Y)$	
\item $V^c(p) = \{ X \in W^c \mid p \in X \}$, то есть, $X \models p \Leftrightarrow p \in X$
\end{itemize}
\end{block}
}

\frame{

\begin{block}{}\alert{Определение}. Последовательность  $\langle X_1, G_1, X_2, \dots, G_{n-1}, X_n \rangle$ т.ч. $X_1, \dots, X_n \in W^c$, $\emptyset \not = G_i \subseteq Ag $, $X_{i} R^{c}_{G_i} X_{i+1}$
будем называть \textbf{каноническим путем}.  
\end{block}

\begin{block}{}\alert{Определение}. Пусть  $\vec{X} = \langle X_1, G_1, X_2, \dots, G_{n-1}, X_n \rangle$, тогда $last(\vec{X})= X_n$.
\end{block}

\begin{block}{}\alert{Определение}. \textbf{$\vec{X}\hat{R}_G\vec{Y}$} $\Leftrightarrow \vec{Y} = \langle \vec{X}, G, Y \rangle$. для некотороко $Y \in W^C$.  То есть, $\vec{Y}$ продолжает $\vec{X}$ на $G$-ребро.
\end{block}
}

\frame{\frametitle{Каноническая модель}
\begin{block}{}\alert{Определение}. $\vec{M} =(\vec{W}, (\vec{R}_i)_{i \in Ag}, \vec{V})$ – каноническая модель, где	
\begin{itemize}
\item $\vec{W} = \{\vec{X} \mid \vec{X} \text{ – путь на } W^c \}$ 
\item $\vec{X}\vec{R}_i\vec{Y} 
\Leftrightarrow \exists G (i \in G \wedge \vec{x}\hat{R}_G\vec{y}) $
\item $\vec{V}(p) = \{ \vec{X} \in \vec{W} \mid p \in last (\vec{X}) \}$, то есть,  
$\vec{M}, \vec{X} \models p \Leftrightarrow  p \in last(\vec{X})$ 
\end{itemize}
\end{block}
}

\frame{\frametitle{Лемма об истинности}
}

\frame{\frametitle{Случай $D_G \varphi$}
}

\frame{\frametitle{Сборка доказательства}
}

\end{document}