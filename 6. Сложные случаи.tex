%----------------------------------------------------------------------------------------
%	PACKAGES AND THEMES
%----------------------------------------------------------------------------------------
\documentclass[aspectratio=169,xcolor=dvipsnames]{beamer}
\usetheme{SimplePlus}

\usepackage[utf8]{inputenc} 
\usepackage[T2A]{fontenc} 
\usepackage[russian]{babel}
\usepackage{amsmath,amssymb}
\usepackage{hyperref}
\usepackage{graphicx} % Allows including images
\usepackage{booktabs} % Allows the use of \toprule, \midrule and \bottomrule in tables
\usepackage{multicol}
\usepackage{expl3,xparse}
\usepackage{ebproof}
\usepackage{mathtools} 
\usepackage{extarrows} 
\usepackage{fitch}
\theoremstyle{plain}
\newtheorem{mydef}{Определение}
\setbeamersize{text margin left=15pt,text margin right=15pt}

%Сокращения для названий языков
\newcommand{\EL}{\mathcal{EL}}
\newcommand{\ELC}{\mathcal{EL\text{-}C}}
\newcommand{\ELD}{\mathcal{EL\text{-}D}}
\newcommand{\ELCD}{\mathcal{EL\text{-}CD}}
\newcommand{\ELE}{\mathcal{EL\text{-}E}}
\newcommand{\PAL}{\mathcal{PAL}}
\newcommand{\PALC}{\mathcal{PAL\text{-}C}}
\newcommand{\PALD}{\mathcal{PAL\text{-}D}}
\newcommand{\PALCD}{\mathcal{PAL\text{-}CD}}
\newcommand{\ELRC}{\mathcal{EL\text{-}RC}}
\newcommand{\PALRC}{\mathcal{PAL\text{-}RC}}

%----------------------------------------------------------------------------------------
%	TITLE PAGE
%----------------------------------------------------------------------------------------
\subtitle{Мини-курс <<Эпистемическая логика: исчисления и модели>>}

\author[]{Виталий Долгоруков, Елена Попова}

\institute[] % Your institution as it will appear on the bottom of every slide, may be shorthand to save space
{Международная лаборатория логики, лингвистики и формальной философии НИУ ВШЭ}
\date{Летняя школа <<Логика и формальная философия>> \\ Факультет свободных искусств и наук \\ 
сентябрь 2022} % Date, can be changed to a custom date


%----------------------------------------------------------------------------------------
%	PRESENTATION SLIDES
%----------------------------------------------------------------------------------------

\usepackage{expl3, xparse}
\usepackage{ebproof}
\title[short title]{Сложные случаи: 
$\PALC$, $\ELRC$, $\PALRC$
}

\begin{document}  \frame{\titlepage}
\section{$PAL$-$C$}

\frame{\frametitle{Сложные случаи}
\begin{itemize}
\item Как аксиоматизировать $\PALC$? 
\item Как связаны $[!\varphi]$ и $C_G\psi$
\item Есть ли аксиома редукции для $[!\varphi]\psi$?
\end{itemize}

}

\frame{\frametitle{$PALC$}
\begin{block}{} \textbf{Утверждение 1}: \alert{Формула $[!\varphi]C_G \psi \leftrightarrow (\varphi \to  C_G [!\varphi] \psi)$ не является общезначимой}.
\end{block}

\begin{proof}
Рассмотрим модель $M, x$
\begin{enumerate}
\item $M, x \models [!p]C_{ab} q$
\item $M, x \models p$
\item $M, x \not \models C_{ab} [!p] q $, поскольку 
$M, x \models \hat{K}_a \hat{K}_b \langle !p \rangle \neg q$
\end{enumerate}	
\end{proof}

}



\frame{\frametitle{Пути}

\begin{block}{} \textbf{Определение 1}. Пусть $M=(W, (R_i)_{i \in Ag}, V)$ – модель Крипке, $x, y \in W$, $G\subseteq Ag$, $x, y \in W$, $G\subseteq Ag$, будем говорить, что существует \alert{$G$-путь} из $x$ в $y$ (обозначение: \alert{$xR_Gy$}), если найдутся такие $y_1, \dots, y_n \in W$ и $i_1, \dots, i_n \in G$, что \alert{$xR_{i_1}y_1R_{i_2} \dots R_{i_n}y_n=y$}.
\end{block}
}

\frame{\frametitle{Что значит $[!\varphi]C_G\psi$?}

\begin{block}{}\textbf{Упражнение 1}. Докажите, что 
	$\big(\bigcup \limits_{i \in G} R^{!\varphi}_i\big)^+  = \; R_{G, \varphi}$
\end{block}

}


\frame{\frametitle{Что значит $[!\varphi]C_G\psi$?}
\begin{block}{}\textbf{Утверждение 1:}
$M, x \models [!\varphi]C_G\psi$ е.т.е. $\forall y (x R_{G, \varphi}y \Rightarrow M, y \models [!\varphi]\psi)$	
\end{block}

\alert{$\blacktriangleright$}   
\begin{enumerate}[<+->] 
\item $M, x \models [!\varphi]C_G\psi \iff$ 
\item  $M, x \models \varphi \Rightarrow M^{!\varphi}, x \models  C_G\psi \iff$ 
\item $M, x \models \varphi \Rightarrow \forall y (x\big(\bigcup \limits_{i \in G} R^{!\varphi}_i\big)^*y \Rightarrow M^{!\varphi}, y \models \psi) \iff$ 
\item  $M, x \models \varphi \Rightarrow \forall y (xR_{G, \varphi}y \Rightarrow M^{!\varphi}, y \models \psi) \iff$ 
\item  $\forall y (M, x \models  \varphi \Rightarrow (xR_{G, \varphi}y \Rightarrow M^{!\varphi}, y \models \psi)) \iff$ 
\item  $\forall y ((M, x \models  \varphi \wedge xR_{G, \varphi}y) \Rightarrow M^{!\varphi}, y \models \psi)) \iff$ 
\item  $\forall y ( xR_{G, \varphi}y  \Rightarrow M^{!\varphi}, y \models \psi) \iff$ 
\item  $\forall y ( (xR_{G, \varphi}y \wedge M, y \models \varphi) \Rightarrow M^{!\varphi}, y \models \psi) \iff $ 
\item  $\forall y (xR_{G, \varphi}y \Rightarrow ( M, y \models \varphi \Rightarrow M^{!\varphi}, y \models \psi)) \iff$ 
\item  $\forall y (xR_{G, \varphi}y \Rightarrow M, y \models [!\varphi]\psi)$
\end{enumerate}
  \alert{$\blacktriangleleft$} 
}

\frame{\frametitle{Публичные объявления и общее знание}
\begin{block}{Лемма}
\begin{center}
\begin{prooftree}
\hypo{\models \chi \to [!\varphi]\psi} 
\hypo{\models (\chi \wedge \varphi) \to E_G \chi}
\infer2{\models \chi \to [!\varphi]C_G\psi}	
\end{prooftree}
\end{center}
\end{block}
}


\frame{
\resizebox{.95\textwidth}{!}{ 
\begin{fitch}
(a) \models \chi \to [!\varphi] \psi \text{, }(b) \models (\chi \wedge \varphi) \to E_G \chi \\
\fh \fw{M,x} \; M, x \models \chi & $\rhd \; M, x \models [!\varphi]C_G \psi \Leftrightarrow \rhd \; \forall y (x R_{G, \varphi}y \Rightarrow M, y \models [!\varphi]\psi)  $ \\
\fa \fh \fw{y}\; x R_{G, \varphi}y & $\rhd \; M, y \models [!\varphi]\psi$ \\ 
\fa \fa xR_{i_1}y_1R_{i_2} \dots R_{i_n}y_n=y \text{ т.ч. } \\
\fa \fa i_1, \dots, i_n \in G \text{ и } M, x, y_1, \dots, y_n \models \varphi & из 4 по опр. $x R_{G, \varphi}y$\\
\fa \fa M, x \models \chi \wedge \phi & 2, 3 \\
\fa \fa M, x \models E_G \chi & 1b, 6 по MP \\
\fa \fa M, x \models K_{i_1} \chi & из 7 т.к. $i_1 \in G$ \\
\fa \fa M, y_1 \models  \chi  & из 5, 8 \\
\fa \fa \vdots & повторяем 6--9 для $y_2, \dots, y_n$ \\
\fa \fa M, y \models  \chi & из 10 \\
\fa \fa M, y \models [!\varphi] \psi & 1, 11 \\
\fa \forall y (x R_{G, \varphi}y \Rightarrow M, y \models [!\varphi]\psi) & В$\forall\Rightarrow$ 4--12 \\
\fa M, x \models [!\varphi]C_G \psi & def \\
\models \chi \to [!\varphi]C_G \psi & В$\forall\Rightarrow$ 2--14
\end{fitch}}
}

\frame{
\begin{exampleblock}{Упражнение}
Переписать предыдущее доказательство в более строгом виде: индукцией по $n$ 
в пункте $xR_{i_1}y_1R_{i_2} \dots R_{i_n}y_n=y $ и далее.
\end{exampleblock}

}

\frame{\frametitle{Исчисление $PALC$}
\begin{itemize}
\item Аксиомные схемы $S5_m$–$C$
\item $[!\varphi]p \leftrightarrow (\varphi \to p)$	
\item $[!\varphi]\neg \psi  \leftrightarrow (\varphi \to \neg [!\varphi]\psi)$
\item $[!\varphi](\psi \wedge \chi) \leftrightarrow ([!\varphi]\psi \wedge [!\varphi]\chi) $
\item $[!\varphi]K_i \psi \leftrightarrow (\varphi \to K_i [!\varphi] \psi)$
\item $[!\varphi][!\psi]\chi \leftrightarrow [!(\varphi \wedge [!\varphi]\psi)]\chi$
\item Правила вывода: $MP$, $NEC$ для $K_i$
\item Правило вывода:
\begin{center}
\begin{prooftree}
\hypo{\chi \to [!\varphi]\psi} 
\hypo{(\chi \wedge \varphi) \to E_G \chi}
\infer2{\chi \to [!\varphi]C_G\psi}	
\end{prooftree}
\end{center}

\end{itemize}


}

\frame{\frametitle{Полнота и корректность}
\begin{block}{Теорема о полноте: схема доказательства}
\begin{itemize}
\item Замыкание $cl(\varphi)$
\item $c(\varphi)$ переопределить
\item Лемма об истинности: случай $[!\varphi]C_G\psi$	
\item Лемма об истинности собирается индукцией по $c$
\end{itemize}
\end{block}
}

\frame{
\begin{block}{}\textbf{Определение}. \alert{Замыкание $cl(\varphi')$}.
Для $\varphi' \in \PALC$ определим четыре множества:
$cl_1(\varphi') \subset cl_2(\varphi') \subset cl_3 (\varphi') \subset cl(\varphi')$.
\begin{description}
\item[$cl_1(\varphi)$] – наименьшее множество, замкнутое по следующим правилам:
\begin{enumerate}
\item $\varphi \in cl_1(\varphi)$
\item если $\psi \in cl_1(\varphi)$, то $Sub(\psi) \subset cl_1(\varphi)$
\item если $C_G \psi \in cl_1(\varphi)$, то $\{K_i C_G \psi \mid i \in G \} \subseteq cl_1(\varphi)$	
\item если $[!\varphi]p \in cl_1(\varphi')$, то $\varphi \to p \in cl_1(\varphi')$
\item если $[!\varphi]\neg\psi \in cl_1(\varphi')$, то $\varphi \to \neg [!\psi]\varphi \in cl_1(\varphi')$
\item если $[!\varphi](\psi \wedge \chi) \in cl_1(\varphi')$, то $[!\varphi] \psi \wedge [!\varphi] \chi \in cl_1(\varphi')$
\item если $[!\varphi]K_i\psi \in cl_1(\varphi')$, то $\varphi \to K_i [!\psi]\varphi \in cl_1(\varphi')$
\item если $[!\varphi][!\psi]\chi \in cl_1(\varphi')$, то $ [!(\varphi \wedge [!\varphi]\psi)]\chi \in cl_1(\varphi')$
\item если $[!\varphi]C_G\psi \in cl_1(\varphi')$, то $[!\varphi]\psi \in cl_1(\varphi')$ и $\{K_i [!\varphi]C_G\psi \mid i \in G   \} \subset cl_1(\varphi') $
\end{enumerate}
\item[$cl_2(\varphi)$]$:= cl_1(\varphi) \cup \{ \neg \psi \mid \psi \in cl_1(\varphi) \text{ и } \psi \not = \neg \dots \}$	 
\item[$cl_3(\varphi)$]$:= cl_2(\varphi) \cup \{K_i K_i \psi \mid K_i \psi \in cl_2(\varphi) \} \cup \{K_i \neg K_i \psi \mid \neg K_i \psi \in cl_2(\varphi) \}$	
\item[$cl(\varphi)$]$:=cl_3(\varphi) \cup \{ \neg \psi \mid \psi \in cl_3(\varphi) \text{ и }  \psi \not = \neg \dots \}$
\end{description}
\end{block}
}


\frame{\frametitle{Сложность $c(\varphi)$}

\begin{block}{} Определение. Определим функцию сложности $c: \PALC \mapsto \mathbb{N}$:
\begin{enumerate}
	\item $c(p):= 1$
	\item $c(\neg \varphi):= c(\varphi) + 1$
	\item $c(\varphi \wedge \psi) = max\{c(\varphi),c(\psi)\} + 1$
	\item $c(K_i \varphi):= c(\varphi) + 1$
	\item $c(C_G \varphi):= c(\varphi) + 1$
	\item $c([!\varphi]\psi):= (c(\varphi) + 4) \cdot c(\psi)$
\end{enumerate}	
\end{block}
}

\frame{
\begin{block}{Лемма}
\begin{itemize}
\item $c(\varphi) \geq c(\psi) $ для $\psi \in sub(\varphi)$ 	
\item $c([!\varphi]p) > c (\varphi \to p)$	
\item $c([!\varphi]\neg \psi)  > c(\varphi \to \neg [!\varphi]\psi)$
\item $c([!\varphi](\psi \wedge \chi)) > c([!\varphi]\psi \wedge [!\varphi]\chi) $
\item $c([!\varphi]K_i \psi) >  c(\varphi \to K_i [!\varphi] \psi)$
\item $c([!\varphi][!\psi]\chi) > c([!(\varphi \wedge [!\varphi]\psi)]\chi)$
\item $c([!\varphi]C_G \psi) > c([!\varphi]\psi) $
\end{itemize}

	
\end{block}

}

\frame{
\begin{block}{}
Утверждение: $\vdash [!\varphi]C_G\psi \to (\varphi \to  K_i[!\varphi] C_G \psi)$ для $i \in G$
\end{block}
\begin{enumerate}
\item $C_G\psi \to K_i C_G\psi $ 
\item $[!\varphi]C_G\psi \to [!\varphi]K_i C_G\psi $
\item $[!\varphi]K_i C_G\psi \to (\varphi \to  K_i[!\varphi] C_G \psi)$
\item $[!\varphi]C_G\psi \to (\varphi \to  K_i[!\varphi] C_G \psi)$
\end{enumerate}	
}

\frame{\frametitle{Лемма об истинности}
\begin{block}{Лемма}
Пусть $\Phi$ – замыкание некоторой формулы, $M^\Phi = (W^\Phi, (R^\Phi_i)_{i \in Ag}, V^\Phi)$ – конечная каноническая модель, $X \in W^\Phi$ тогда
$$\alert{\forall \varphi' \in \Phi: \varphi' \in X \iff M^\Phi, X \models \varphi'}$$
\end{block}
}

\frame{
\begin{block}{Доказательство}
Будем доказывать (возвратной) индукцией по $c(\varphi')$.
\begin{description}
\item[Предположение индукции] Обозначим $c(\varphi')=n$.
$\forall \psi\in \Phi: c(\psi) < n \Rightarrow ( \psi \in X \iff M^\Phi, X \models \psi) $
\item[Шаг индукции] Рассмотрим следующие случаи. 

\begin{itemize}
\item[Сл.1] $\varphi'= p$
\item[Сл.2] $\varphi'= \neg \varphi $
\item[Сл.3] $\varphi'= \varphi \wedge \psi$
\item[Сл.4] $\varphi'= K_i \varphi $
\item[Сл.5] $\varphi'= C_G \varphi $
\item[Сл.6] $\varphi'= [\varphi]\psi$
   \begin{itemize}
     \item[Сл.6a] $\varphi'= [\varphi]p $ 
     \item[Сл.6b] $\varphi'= [\varphi]\neg \psi$ 
     \item[Сл.6c] $\varphi'= [\varphi](\psi \wedge \chi)$ 
     \item[Сл.6d] $\varphi'= [\varphi]K_i \psi$ 
     \item[Сл.6e] $\varphi'= [\varphi][\psi]\chi$ 
     \item[Сл.6f] $\varphi'= [\varphi]C_G \psi p$ 
   \end{itemize}
\end{itemize}
\end{description}
\end{block}
}

\frame{\frametitle{Сл. 6a-6e}
Сл.6a \\
$c(\varphi \to p)<c([!\varphi]p)$ \\
$[!\varphi]p \in X \xLeftrightarrow{} (\varphi \to p)\in X
\xLeftrightarrow{} M^\Phi, X \models \varphi \to p 
\xLeftrightarrow{} M^\Phi, X \models [!\varphi]p
$	\\
Сл.6b–d. Упражнение. \\
 Сл.6e \\  
$[!\varphi][!\psi]\chi \in X \xLeftrightarrow[\text{Акс}]{\Phi}
[!(\varphi \wedge [!\varphi]\psi)]\chi \in X \xLeftrightarrow[(*)]{\text{IH}}
M^\Phi, X \models [!(\varphi \wedge [!\varphi]\psi)]\chi \xLeftrightarrow{\text{}}
M^\Phi, X \models 
[!\varphi][!\psi]\chi$ \\ 
$(*) \;  c([!(\varphi \wedge [!\varphi]\psi)]\chi) < c([!\varphi][!\psi]\chi)$ 
}

\frame{\frametitle{Случай 6f$\Rightarrow$}
\resizebox{0.95\textwidth}{!}{ 
\begin{fitch}
\fh [!\varphi]C_G\psi \in X & $\rhd \; M^\Phi, X \models [!\varphi]C_G\psi \Leftrightarrow \rhd \; \forall Y (X R_{G, \varphi} Y \Rightarrow M^\Phi, Y \models [!\varphi]\psi )$\\
\fa \fh \fw{Y} \; X R^{\Phi}_{G,\varphi}Y & $\rhd \; M^\Phi, Y \models [!\varphi]\psi$ \\
\fa \fa XR^{\Phi}_{i_1}Y_1R^{\Phi}_{i_2} \dots R^{\Phi}_{i_n}Y_n=Y \text{ т.ч. } i_1, \dots , i_n \in G & из 2 по опр. \\
\ftag{~}{\fa \fa  \text{и } M^\Phi, X \models \varphi, M^\Phi, Y_1 \models \varphi, \dots,  M^\Phi, Y_n \models \varphi}  \\ 
\fa \fa \varphi \in X, \varphi \in Y_1, \dots, \varphi \in Y_n & ПИ\\
\fa \fa \varphi \to  K_i[!\varphi] C_G \psi \in X & по утв. на сл. 10 и $\varphi \to  K_i[!\varphi] C_G \psi \in X \in \Phi$ \\
\fa \fa K_{i_1}[!\varphi] C_G \psi \in X & из 4,5 по MP \\
\fa \fa XR^{\Phi}_{i_1}Y_1 & из 3\\
\fa \fa [!\varphi] C_G \psi \in Y_1 & из 6,7 по опр. \\
\fa \fa \vdots & повторяем шаги 5--8 для  $Y_2$ и т.д. до  $Y_n=Y$ \\
\fa \fa [!\varphi] C_G \psi \in Y & из 9 \\
\fa \fa [!\varphi] \psi \in Y & из 10, $\vdash C_G \psi \to [!\varphi]  \psi $ и $[!\varphi]  \psi \in \Phi$ \\
\fa \fa  M^\Phi, Y \models [!\varphi] \psi & ПИ\\
\fa \forall Y (XR^{\Phi}_{G,\varphi}Y \Rightarrow  [!\varphi]\psi \in Y) & 2--11 В$\forall\Rightarrow$
\end{fitch}
}
}


%\frame{\frametitle{$[!\varphi]C_G\psi \in X \Rightarrow M^{\Phi}, X \models [!\varphi]C_G\psi$}
%
%\begin{block}{Что мы уже доказали?}
%\begin{enumerate}
%	\item $M, x \models [!\varphi]C_G\psi$ е.т.е. $\forall y (x R_{G, \varphi}y \Rightarrow M, y \models [!\varphi]\psi)$
%	\item $[!\varphi]C_G\psi \in X \Rightarrow \forall Y (X R^{\Phi}_{G,\varphi}Y \Rightarrow  [!\varphi]\psi \in Y)$
%\end{enumerate}	
%\end{block}
%
%
%\begin{fitch}
%[!\varphi]C_G\psi \in X \\
%\forall Y (X R^{\Phi}_{G,\varphi}Y \Rightarrow  [!\varphi]\psi \in Y) \\
%\forall Y (X R^{\Phi}_{G,\varphi}Y \Rightarrow  M^{\Phi}, Y \models [!\varphi]\psi ) &  по ПИ \\
%M^{\Phi}, X \models [!\varphi]C_G\psi
%\end{fitch}
%}

\frame{\frametitle{Случай 6f($\Leftarrow$): сборка доказательства}
Упражнение
}

\frame{
\begin{block}{}\textbf{Утверждение}.
$\vdash \chi \to [!\varphi]\psi$ 
\end{block}
Достаточно доказать, что $\underline{X} \to [!\varphi]\psi$ для $X \in S$.
\begin{enumerate}
\item $M^\Phi, X \models [!\varphi]C_G \psi$	
\item $M^\Phi, X \models [!\varphi] \psi$
\item $c([!\varphi]\psi) < c([!\varphi]C_G\psi)$
\item $[!\varphi] \psi \in X $ по П.И.
\item $X \vdash [!\varphi] \psi $ 
\item $\vdash \underline{X} \to [!\varphi] \psi $
\end{enumerate}
}

\frame{\frametitle{Случай 6f$\Leftarrow$}
\begin{block}{Лемма $(\chi \wedge \varphi) \to E_G \neg \underline{Y}$}
Достаточно доказать, для любых $X \in S, Y \in \overline{S}, i \in G$ $\vdash (\underline{X} \wedge \varphi) \to K_i \neg \underline{Y} $	
\end{block}
\resizebox{1.0\textwidth}{!}{ 

\begin{fitch}
X \in S \\
Y \in \overline{S} \\ 
\fh \not \vdash (\underline{X} \wedge \varphi) \to K_i \neg \underline{Y} &  $\rhd \metabot$  \\
\fa \underline{X}, \varphi, \neg K_i \neg \underline{Y} \not \vdash \bot \\
\fa X, \varphi, \hat{K}_i \underline{Y} \not \vdash \bot \\
\fa X, \varphi \not \vdash \bot \\
\fa \varphi \in X \\
\fa X, \hat{K}_i \underline{Y} \not \vdash \bot 
\end{fitch}

%\ftag{9}{\fa Y \vdash \neg  \psi   } \setcounter{fitchcounter}{9} 

\begin{fitch}
\ftag{9}{\fa X R^\Phi_i Y}  \setcounter{fitchcounter}{9} \\
\fa M^\Phi, X \models \varphi & из 6 по ПИ \\
\fa \models [!\varphi] C_G \psi \to (\varphi \to K_i [!\varphi] C_G \psi) \\
\fa M^\Phi, X \models \varphi \to K_i [!\varphi] C_G \psi \\
\fa M^\Phi, X \models K_i [!\varphi] C_G \psi \\
\fa M^\Phi, Y \models [!\varphi] C_G \psi \\
\fa Y \in S \\
\fa \metabot  & 1, 14
\end{fitch} }
}

%\frame{\frametitle{Как быть с $RE$?}
%$[!\varphi][!\psi]\chi$
%}


\section{Условное общее знание}
\subsection{$\ELRC$}

\frame{\frametitle{Аксиомы редукции}
%TOD напомнить языки
\begin{itemize}
\item $\EL \equiv \PAL$
\item $\ELD \equiv \PALD$
\item $\ELC \prec \PALC$
\item $\ELC$+? $\equiv \PALC$ +?
\item $\ELRC \equiv \PALRC$
\end{itemize}


}
\frame{\frametitle{Условное общее знание}

\begin{block}{}\textbf{Определение 3}.
$M, x \models \alert{C^{\psi}_G \varphi}$ е.т.е.
$\forall y (x\big(\bigcup \limits_{i \in G} \! R_i \cap \; (W \times [\psi]_M)  \big)^+y \Rightarrow M, y \models \varphi )$	
\end{block}


\begin{block}{} \textbf{Утверждение}: Общее знание выразимо через условное общее знание:
\begin{center}
	$C_G \varphi \equiv C^{\top}_G \varphi$
\end{center}	
Доказательство: упражнение
\end{block}
}

\frame{

\begin{block}{} \textbf{Определение 2}.
Пусть $M=(W, (R_i)_{i \in Ag}, V)$ – модель Крипке, $x, y \in W$, $G\subseteq Ag$, будем говорить, что существует \alert{$G$-$\varphi$-путь} из $x$ в $y$ (обозначение: \alert{$xR_{G, \varphi}y$}), если  найдутся такие $y_1, \dots, y_n \in W$ и $i_1, \dots, i_n \in G$, что   \alert{$xR_{i_1}y_1R_{i_2} \dots R_{i_n}y_n=y$} и \alert{$M, x \models \varphi, M, y_1 \models \varphi, \dots, M, y_n \models \varphi$}.	
\end{block}

\begin{block}{} \textbf{Определение 3}.
Пусть $M=(W, (R_i)_{i \in Ag}, V)$ – модель Крипке, $x, y \in W$, $G\subseteq Ag$, будем говорить, что существует \alert{$G$-$\cdot\varphi$-путь} из $x$ в $y$ (обозначение: \alert{$R_{G,\cdot\varphi}$}), если  найдутся такие $y_1, \dots, y_n \in W$ и $i_1, \dots, i_n \in G$, что   \alert{$xR_{i_1}y_1R_{i_2} \dots R_{i_n}y_n=y$} и \alert{$M, y_1 \models \varphi, \dots, M, y_n \models \varphi$}.	
\end{block}

\begin{block}{}
	$M, x \models C^{\psi}_G \varphi$ е.т.е. $\forall y (xR_{G, \cdot \psi}y \Rightarrow M, y \models \varphi)$ 
\end{block}


}

\frame{\frametitle{Исчисление для условного общего знания}

\begin{block}{Исчисление $S5_m$–$RC$}
Аксиомные схемы:
\begin{description}
\item[($S5_{K}$)] Аксиомные схемы $S5$ для $K_i$
\item[($K_{RC}$)] $C^\chi_G(\varphi \to \psi) \to (C^\chi_G \varphi \to C^\chi_G \psi)$ 
\item[($mix_{RC}$)] $C^\psi_G \varphi \leftrightarrow E_G(\psi \to (\varphi \wedge C^\psi_G \varphi) )$ 
\item[($ind_{RC}$)] $C^\psi_G (\varphi \to E_G(\psi \to \varphi)) \to (E_G(\psi \to \varphi) \to C^\psi_G \varphi)$
\end{description}

Правила вывода: 
\begin{multicols}{3}
\begin{prooftree}
\hypo{\varphi} 
\hypo{\varphi \to \psi }
\infer2[$MP$]{\psi}	
\end{prooftree}

\begin{prooftree}
\hypo{\varphi} 
\infer1[$G_K$]{K_i \varphi}	
\end{prooftree}

\begin{prooftree}
\hypo{\varphi} 
\infer1[$G_CK$]{C^\psi_G \varphi}	
\end{prooftree}
\end{multicols}
\end{block}
}

\frame{\frametitle{Некоторые полезные теоремы}
\begin{block}{}\textbf{Упражнение}.Найдите доказательства для следующих теорем исчисления $S5_m$–$RC$:
\begin{enumerate}
\item $C^\psi_G \varphi \to E_G (\psi \to \varphi)$
\item $C^\psi_G \varphi \to E_G (\psi \to E_G (\psi \to \varphi))$
\item $C^\varphi_G \varphi $
\item $C^\psi_G \varphi \to C^\psi_G C^\psi_G \varphi$
\item $C^\psi_G \varphi \leftrightarrow C^\psi_G(\psi \wedge  \varphi)$
\item $C^\psi_G \varphi \leftrightarrow C^\psi_G (\psi \to \varphi)$
\end{enumerate}
\end{block}
}

\frame{\frametitle{Полнота $S5_m$-$RC$}
Сборка доказательства. 
\begin{itemize}
\item $\Phi = cl(\varphi)$
\item Лемма об истинности: случай $C^\psi_G\varphi$	
\end{itemize}
}


\frame{\frametitle{Замыкание}
\begin{block}{}\textbf{Определение}. \alert{Замыкание $cl(\varphi')$}.
Для $\varphi' \in \PALRC$ определим четыре множества:
$cl_1(\varphi') \subset cl_2(\varphi') \subset cl_3 (\varphi') \subset cl(\varphi')$.
\begin{description}
\item[$cl_1(\varphi)$] – наименьшее множество, замкнутое по следующим правилам:
\begin{enumerate}
\item $\varphi \in cl_1(\varphi')$
\item если $\psi \in cl_1(\varphi')$, то $Sub(\psi) \subset cl_1(\varphi')$
\item если $C_G \psi \in cl_1(\varphi')$, то $\{K_i C_G \psi \mid i \in G \} \subseteq cl_1(\varphi')$	
\item  если $C^\psi_G \varphi \in cl_1(\varphi')$, то $\{K_i (\psi \to (\varphi \wedge C^\psi_G \varphi ) \mid i \in G \} \subset cl_1(\varphi')$
\end{enumerate}
\item[$cl_2(\varphi)$]$:= cl_1(\varphi) \cup \{ \neg \psi \mid \psi \in cl_1(\varphi) \text{ и } \psi \not = \neg \dots \}$	 
\item[$cl_3(\varphi)$]$:= cl_2(\varphi) \cup \{K_i K_i \psi \mid K_i \psi \in cl_2(\varphi) \} \cup \{K_i \neg K_i \psi \mid \neg K_i \psi \in cl_2(\varphi) \}$	
\item[$cl(\varphi)$]$:=cl_3(\varphi) \cup \{ \neg \psi \mid \psi \in cl_3(\varphi) \text{ и }  \psi \not = \neg \dots \}$
\end{description}
\end{block}
}

\frame{\frametitle{Лемма об истинности: Cл. $\psi'= C^\psi_G\varphi$ ($\Rightarrow$)}
%\begin{block}{Cлучай $C^\psi_G\varphi$. ($\Rightarrow$)$C^\psi_G \varphi \in \Phi$}
%Для $C^\psi_G\varphi \in X \Rightarrow M^\Phi, X \models C^\psi_G \varphi$
%\end{block}
\resizebox{0.9\textwidth}{!}{ 
\begin{fitch}
\fh C^\psi_G\varphi \in X & $\rhd \; M^\Phi, X \models C^\psi_G \varphi \Leftrightarrow \rhd \; \forall Y (X(R^{\Phi}_{G, + \psi}) Y \Rightarrow M^\Phi, Y \models \varphi)$ \\
\fa \fh  \fw{Y} \; X(R^{\Phi}_{G, + \psi}) Y & $\rhd \; M^\Phi, Y \models \varphi$ \\
\fa \fa XR^\Phi_{i_1}Y_1R^\Phi_{i_2}\dots R^\Phi_{i_n}Y_n = Y \text{ т.ч.} & \\
\fa \fa \{i_1, \dots, i_n \} \subseteq G \\
\fa \fa M^\Phi, Y_1 \models \psi, \dots, M^\Phi, Y_n \models \psi \\
\fa \fa \psi \in Y_1, \dots, \psi \in Y_n & по ПИ \\
\fa \fa K_{i_1}(\psi \to (\varphi \wedge C^\psi_G \varphi)) \in X \\
\fa \fa  \psi \to (\varphi \wedge C^\psi_G \varphi) \in Y_1 \\
\fa \fa \varphi \wedge C^\psi_G \varphi \in Y_1 \\
\fa \fa \text{повторяем до } Y_n=Y \\
\fa \fa \varphi \wedge C^\psi_G \varphi \in Y_n \\
\fa \fa \varphi \in Y_n \\
\fa \fa  M^\Phi, Y \models \varphi & по ПИ
\end{fitch}}
}

\frame{\frametitle{Лемма об истинности: Cл. $\psi'= C^\psi_G\varphi$. ($\Leftarrow$)}
\begin{itemize}
\item $\underline{X} \to E_G(\psi \to \chi)$
\item $\chi \to E_G(\psi \to \chi)$
\item $\chi \to (\psi \to \varphi)$
\end{itemize}
}


\frame{\frametitle{Cл. $\psi'= C^\psi_G\varphi$ ($\Leftarrow$) Сборка доказательства}
\begin{center}
$\footnotesize\boxed{S:= \{ X \in W^\Phi \mid  M^\Phi, X \models C^\psi_G \varphi  \}}$ 
$\footnotesize\boxed{\chi:= \bigvee \{\underline{X} 
\mid  X \in S \} }$
$\footnotesize\boxed{\overline{S}:= W^\Phi \backslash S}$
\end{center}
\begin{center}
\resizebox{1.02\textwidth}{!}{
\begin{prooftree}[separation=0.9em]
\hypo{\underline{X} \to E_G(\psi \to \chi)} 
    \hypo{\chi \to E_G(\psi \to \bigwedge \{\neg \underline{Y} \mid Y \in \overline{S} \})} \hypo{\chi \leftrightarrow \bigwedge \{\neg \underline{Y} \mid Y \in \overline{S} \}}
       \infer2{\chi \to E_G(\psi \to \chi)}
       \infer1{C^\psi_G(\chi \to E_G(\psi \to \chi))}         \hypo{C^\psi_G(\chi \to E_G(\psi \to \chi)) \to (E_G(\psi \to \chi) \to C^\psi_G \chi)}
       \infer2{E_G(\psi \to \chi) \to C^\psi_G \chi}
\infer2{\underline{X} \to C^\psi_G \chi}
                           \hypo{C^\psi_G\chi \to C^\psi_G(\psi \wedge \chi)}
\infer2{\underline{X} \to C^\psi_G(\psi \wedge \chi)}

\hypo{(\psi \wedge \chi) \to \varphi}
\infer1{C^\psi_G(\psi \wedge \chi) \to C^\psi_G\varphi}

\infer2{\underline{X} \to C^\psi_G \varphi}
\infer1{X \vdash C^\psi_G \varphi}
%\hypo{C^\psi_G \varphi \in \Phi}
\infer1{C^\psi_G \varphi \in X}                            	
\end{prooftree}
}
\end{center}

}


\frame{

\begin{center}
$\footnotesize\boxed{S:= \{ X \in W^\Phi \mid  M^\Phi, X \models C^\psi_G \varphi  \}}$ 
$\footnotesize\boxed{\chi:= \bigvee \{\underline{X} 
\mid  X \in S \} }$
$\footnotesize\boxed{\overline{S}:= W^\Phi \backslash S}$
\end{center}

\begin{block}{} \textbf{Утверждение}.
\begin{center}
$\vdash \chi \to E_G (\psi \to \bigwedge \{\neg \underline{Y} \mid Y \in \overline{S} \}) \Leftrightarrow \forall i \in G \; \forall X \in S \; \forall Y \in \overline{S} \vdash \underline{X} \to K_i (\psi \to \neg \underline{Y}) $
\end{center}
\end{block}

Доказательство.
\begin{enumerate}
\item $ \vdash \chi \to E_G (\psi \to \bigwedge \{\neg \underline{Y} \mid Y \in \overline{S} \}) \Leftrightarrow $
\item $\forall i \in G$ $\vdash \chi \to K_i (\psi \to \bigwedge \{\neg \underline{Y} \mid Y \in \overline{S} \}) \Leftrightarrow $
\item $\forall i \in G \; \forall X \in S$ $ \vdash \underline{X} \to  K_i (\psi \to \bigwedge \{\neg \underline{Y} \mid Y \in \overline{S} \}) \Leftrightarrow $ 	
\item $\forall i \in G \; \forall X \in S$  $\vdash \underline{X} \to  K_i (\bigwedge \{\psi \to \neg \underline{Y} \mid Y \in \overline{S} \}) \Leftrightarrow $ 
\item $\forall i \in G \; \forall X \in S$ $\vdash \underline{X} \to  \bigwedge \{K_i (\psi \to \neg \underline{Y}) \mid Y \in \overline{S} \} \Leftrightarrow $
\item $\forall i \in G \; \; \forall X \in S\; \forall Y \in \overline{S}$ $\vdash \underline{X} \to K_i (\psi \to \neg \underline{Y}) $
\end{enumerate}



}

\frame{
\begin{center}
$\footnotesize\boxed{S:= \{ X \in W^\Phi \mid  M^\Phi, X \models C^\psi_G \varphi  \}}$ 
$\footnotesize\boxed{\chi:= \bigvee \{\underline{X} 
\mid  X \in S \} }$
$\footnotesize\boxed{\overline{S}:= W^\Phi \backslash S}$
\end{center}
\begin{block}{}\textbf{Утверждение}. Пусть $X \in S$, $Y \in \underline{S}$, тогда \alert{$\vdash \underline{X} \to K_i(\psi \to \neg \underline{Y})$}
\end{block}
\resizebox{1.0\textwidth}{!}{ 
\begin{fitch}
X \in S, Y \in \underline{S} \\
M^\Phi, X \models  C^\psi_G\varphi \\
M^\Phi, Y \not \models  C^\psi_G\varphi \\
\fh \not \vdash \underline{X} \to K_i(\psi \to \neg \underline{Y}) & $\rhd \; \metabot$ \\
\fa \underline{X} \not \vdash K_i(\psi \to \neg \underline{Y}) \\
\fa \underline{X}\not \vdash \neg K_i(\psi \to \neg \underline{Y}) \to \bot \\
\fa \underline{X}, \neg K_i(\psi \to \neg \underline{Y}) \not \vdash  \bot \\
\fa \underline{X}, \hat{K}_i(\psi \wedge \underline{Y}) \not \vdash  \bot 
\end{fitch}
\begin{fitch}
\ftag{9}{X R^\Phi_i Y}& 8 по утв. (*) \setcounter{fitchcounter}{9} \\
\fa \psi \in Y & 8 по утв. (*) \\
\fa X \models K_i (\psi \to C^\psi_G\varphi) \\
\fa Y \models \psi \to C^\psi_G \varphi \\
\fa Y \models \psi & по ПИ \\ 
\fa Y \models C^\psi_G\varphi \\
\fa \metabot \\
\vdash \underline{X} \to K_i(\psi \to \neg \underline{Y})
\end{fitch}}	

}

\frame{
\begin{block}{}\textbf{Утверждение (*)}. Пусть $X, Y \in W^\Phi$, тогда
\begin{center}
\alert{$\underline{X}, \hat{K}_i(\varphi \wedge \underline{Y}) \not \vdash  \bot \Rightarrow 
(XR^{\Phi}_iY \text{ и } \varphi \in Y)$}
\end{center}
\end{block}

}

\frame{
\begin{block}{}\textbf{Утверждение} $\vdash (\psi \wedge \chi) \to \varphi$
\end{block}
Достаточно доказать, что $\vdash \underline{X} \to (\psi \to \varphi)$ для $X\in S $. \\

\resizebox{1.0\textwidth}{!}{ 
\begin{fitch}
M^\Phi, X \models C^\psi_G	\varphi \\
M^\Phi, X \models K_i (\psi \to \varphi) & для $i \in G$ \\
\fh y_0 = \#_i X \cup \{\psi\} \cup \{ \neg \varphi \} \not \vdash \bot & $\rhd \; \metabot$ \\
\fa y_0 \subset Y \in W^\Phi & по л.Линденбаума \\
\fa XR^\Phi_iY & по опр $R^\Phi$ из 3, 4 \\
\fa M^\Phi, Y  \models \psi \to \varphi \\
\fa M^\Phi, Y  \models \psi \Rightarrow  M^\Phi, Y  \models \varphi \\
\fa \psi \in Y \Rightarrow \varphi \in Y & по ПИ \\
\fa \varphi \in Y 
\end{fitch}

\begin{fitch}
\ftag{10}{ \fa \neg \varphi \in Y} \setcounter{fitchcounter}{10} \\ 
\fa \metabot	\\
\#_i X \cup \{\psi\} \cup \{ \neg \varphi \}  \vdash \bot & 3--11 \\
\#_i X \cup \{\psi\} \vdash \varphi \\
\#_i X \vdash \psi \to \varphi \\
X \vdash \#_i X & по $S5$ \\
X \vdash \psi \to \varphi \\
\vdash \underline{X} \to  (\psi \to \varphi)
\end{fitch}
}


}


\subsection{$\PALRC$}

\frame{\frametitle{Аксиома редукции для условного общего знания}
\begin{block}{Исчисление $S5_m[]$-$RC$ ($PAL$-$RC$)}
\begin{description}
\item[($S5_mRC$)] Аксиомные схемы и правила вывода исчисления $S5_mRC$
\item[($R_{RC}$)] $[!\varphi]C^\chi_G \psi \leftrightarrow 
(\varphi \to C^{\varphi \wedge [!\varphi]\chi}_G[!\varphi]\psi)
 $
\end{description}
\end{block}

\begin{exampleblock}{Упражнение}
Сформулируйте аксиому редукции для общего знания:
$[!\varphi]C_G \psi \leftrightarrow ? $
\end{exampleblock}

\begin{exampleblock}{Упражнение} Для формулы $[!p]C_Gq$ найдите эквивалентную, но из языка $\ELRC$.	
\end{exampleblock}
}

\section{Сравнение языков по выразительной силе}

\frame{\frametitle{Сравнение языков по выразительной силе}
\begin{itemize}
\item $\ELC \prec \PALC$ 
$$[!(\neg p \to K_a \neg p)]C_{ab}\neg p$$
\item $\ELRC \equiv \PALRC$
\item $\PALC \prec \ELRC$
$$ C^{p}_{ab}\neg K_a p$$	
\end{itemize}
Подробнее: [vanDitmarsch2008]
}

\end{document}