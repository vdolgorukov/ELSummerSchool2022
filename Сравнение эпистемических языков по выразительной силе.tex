%----------------------------------------------------------------------------------------
%	PACKAGES AND THEMES
%----------------------------------------------------------------------------------------
\documentclass[aspectratio=169,xcolor=dvipsnames]{beamer}
\usetheme{SimplePlus}

\usepackage[utf8]{inputenc} 
\usepackage[T2A]{fontenc} 
\usepackage[russian]{babel}
\usepackage{amsmath,amssymb}
\usepackage{hyperref}
\usepackage{graphicx} % Allows including images
\usepackage{booktabs} % Allows the use of \toprule, \midrule and \bottomrule in tables
\usepackage{multicol}
\usepackage{expl3,xparse}
\usepackage{ebproof}
\usepackage{mathtools} 
\usepackage{extarrows} 
\usepackage{fitch}
\theoremstyle{plain}
\newtheorem{mydef}{Определение}
\setbeamersize{text margin left=15pt,text margin right=15pt}

%Сокращения для названий языков
\newcommand{\EL}{\mathcal{EL}}
\newcommand{\ELC}{\mathcal{EL\text{-}C}}
\newcommand{\ELD}{\mathcal{EL\text{-}D}}
\newcommand{\ELCD}{\mathcal{EL\text{-}CD}}
\newcommand{\ELE}{\mathcal{EL\text{-}E}}
\newcommand{\PAL}{\mathcal{PAL}}
\newcommand{\PALC}{\mathcal{PAL\text{-}C}}
\newcommand{\PALD}{\mathcal{PAL\text{-}D}}
\newcommand{\PALCD}{\mathcal{PAL\text{-}CD}}
\newcommand{\ELRC}{\mathcal{EL\text{-}RC}}
\newcommand{\PALRC}{\mathcal{PAL\text{-}RC}}

%----------------------------------------------------------------------------------------
%	TITLE PAGE
%----------------------------------------------------------------------------------------
\subtitle{Мини-курс <<Эпистемическая логика: исчисления и модели>>}

\author[]{Виталий Долгоруков, Елена Попова}

\institute[] % Your institution as it will appear on the bottom of every slide, may be shorthand to save space
{Международная лаборатория логики, лингвистики и формальной философии НИУ ВШЭ}
\date{Летняя школа <<Логика и формальная философия>> \\ Факультет свободных искусств и наук \\ 
сентябрь 2022} % Date, can be changed to a custom date


%----------------------------------------------------------------------------------------
%	PRESENTATION SLIDES
%----------------------------------------------------------------------------------------


\begin{document}  \frame{\titlepage}


\section{Сравнение языков}


\frame{\frametitle{Эпистемические языки}
\begin{block}{Языки}
\begin{center}
\begin{tabular}{ll}
$(\EL)$   &    $\varphi, \psi::= p \mid \neg \varphi \mid (\varphi \wedge \psi) \mid K_i \varphi$ \\
$(\ELE)$  &   $\varphi, \psi::=  p \mid \neg \varphi \mid (\varphi \wedge \psi) \mid K_i \varphi \mid E_G \varphi$\\
$(\ELD)$ &   $\varphi, \psi::=  p \mid \neg \varphi \mid (\varphi \wedge \psi) \mid K_i \varphi \mid D_G \varphi$\\
$(\ELC)$	&  $\varphi, \psi::=   p \mid \neg \varphi \mid (\varphi \wedge \psi) \mid K_i \varphi \mid C_G \varphi$
\end{tabular} 
($i \in Ag, G\subseteq Ag$)
\end{center}
\end{block}
}

\frame{\frametitle{Эквивалентность формул}
\begin{block}{Определение}
Будем говорить, что формулы $\varphi$  и $\psi$ эквивалентны в классе моделей $C$ (обозначение: <<$\varphi \equiv_C \psi$>>) е.т.е. 
\begin{center}
$\forall M \in C \forall x \in  D(M): M, x \models \varphi \iff M, x  \models \psi$		
\end{center}
Обозначение: $\varphi \equiv \psi := \varphi \equiv_K \psi$, где $K$ – класс всех моделей Крипке.
\end{block}

\begin{block}{Примеры}
\begin{itemize}
\item $(\Box p \wedge \Box (p \to q)) \equiv \Box (p \wedge q)$	
\item $\Diamond \Box p \equiv_{S5} \Box p$, но $\Diamond \Box p \not \equiv_{S4} \Box p$
\item $\Box \Box p \equiv_{S4} \Box p$, но $\Box \Box p \not \equiv \Box p$
\end{itemize}
\end{block}
}

\frame{
\begin{block}{Определение}
Будем говорить, что формулы $\varphi$  и $\psi$ эквивалентны в классе моделей $C$ (обозначение: <<$\varphi \equiv_C \psi$>>) е.т.е. 
\begin{center}
$\forall M \in C \forall x \in  D(M): M, x \models \varphi \iff M, x  \models \psi$		
\end{center}
Обозначение: $\varphi \equiv \psi := \varphi \equiv_K \psi$, где $K$ – класс всех моделей Крипке.
\end{block}
\begin{block}{Утверждение}
\begin{enumerate}
\item Если $\varphi \equiv_C \psi$ и $C' \subseteq C$, то $\varphi \equiv_{C'} \psi$
\item Если $\varphi \not \equiv_C \psi$ и $C \subseteq C'$, то $ \varphi \not \equiv_{C'} \psi$	
\end{enumerate}

	
\end{block}

}

\frame{\frametitle{Сравнение языков по выразительной силе}
\begin{block}{Определение}
Модальный язык $L_2$ является не менее выразительным, чем модальный язык $L_1$ в классе моделей Крипке $C$ (обозначение: \alert{$L_1 \preceq_{C} L_2$}) е.т.е.  
\begin{center}
$\forall \varphi_1 \in L_1 \exists \varphi_2 \in L_2: \varphi_1 \equiv_C \varphi_2$
\end{center}
\end{block}

\begin{block}{Cокращения}
\begin{itemize}
\item $L_1 \preceq L_2:= L_1 \preceq_{K} L_2$, где $K$ – класс всех моделей Крипке 
\begin{multicols}{2}
\item $L_1 \equiv_{C} L_2 := L_1 \preceq_{C} L_2 \text{ и } L_2  \preceq_{C} L_1 $
\item $L_1 \prec_{C} L_2 := L_1 \preceq_{C} L_2 \text{ и } L_2  \not \preceq_{C} L_1 $
\item $L_1 \equiv  L_2 := L_1 \preceq  L_2 \text{ и } L_2  \preceq  L_1 $
\item $L_1 \prec L_2 := L_1 \preceq L_2 \text{ и } L_2  \not \preceq L_1 $
\end{multicols}
\end{itemize}
\end{block}
}

\frame{\frametitle{Сравнение эпистемических языков}
\begin{block}{Эпистемические языки }
\begin{center}
\begin{tabular}{ll}
$(\EL)$   &    $\varphi, \psi::= p \mid \neg \varphi \mid (\varphi \wedge \psi) \mid K_i \varphi$ \\
$(\ELE)$  &   $\varphi, \psi::=  p \mid \neg \varphi \mid (\varphi \wedge \psi) \mid K_i \varphi \mid E_G \varphi$\\
$(\ELD)$ &   $\varphi, \psi::=  p \mid \neg \varphi \mid (\varphi \wedge \psi) \mid K_i \varphi \mid D_G \varphi$\\
$(\ELC)$	&  $\varphi, \psi::=   p \mid \neg \varphi \mid (\varphi \wedge \psi) \mid K_i \varphi \mid C_G \varphi$
\end{tabular} 
($p \in Var, i \in Ag, G\subseteq Ag$)
\end{center}
\end{block}


\begin{block}{Утверждение 1. }
$\EL \equiv \ELE$
\end{block}

\begin{block}{Утверждение}
$\EL \preceq \ELD$
\end{block}

\begin{block}{Утверждение}
$\EL \preceq \ELC$
\end{block}
}

\frame{\frametitle{Бисимуляция}
}

\frame{\frametitle{$\EL\prec \ELD$}
Пример
}

\frame{\frametitle{Модальная глубина}
\begin{block}{Определение. <<Модальная глубина формулы>> ($md$)}
$md: L_K \mapsto \mathbb{N}$ т.ч. 
\begin{itemize}
\item $md(p)=0$	
\item $md(\neg \varphi) = md(\varphi)$	
\item $md(\varphi \circ \psi ) = max \{md(\varphi), md(\psi)\}$, где $\circ \in \{ \wedge, \vee, \to \}$
\item $md(K_i \varphi) = md(\varphi) + 1$
\end{itemize}
\end{block} 

\begin{block}{Примеры}
\begin{itemize}
	\item $md (K_a K_b p) =  2$ 
	\item $md (K_a K_b (p \to K_c( K_a p \to K_b q)) = 4$
\end{itemize}	
\end{block}
}

\frame{\frametitle{Модальная $n$-эквивалентность}
\begin{block}{Определение. Ограничение языка до глубины $n$}
\centering
$EL_n = \{ \varphi \in EL \mid md(\varphi) \leq n\}$
\end{block}

\begin{block}{Модальная $n$-эквивалентность}
$(M, x) \equiv_n (M', x') \iff \forall \varphi \in EL_n: M, x \models \varphi \Leftrightarrow M', x' \models \varphi$
\end{block}
}

\frame{\frametitle{Пример}
$(M_1, x) \equiv_2 (M_2, x)$, но $(M_1, x) \not \equiv_3 (M_2, x)$ \\
Как доказать? 
}

\frame{\frametitle{n-бисимуляционные игры}
\begin{block}{Определение.}
$BG_{n}[(M, x), (M', x')]$. Игра останавливает после n раундов. В остальном правила такие же как и для $BG_{\infty}[(M, x),(M', x')]$.
\end{block}
\begin{block}{Определение.}
Две модели Крипке эквивалентны в игровом смысле ($(M, x) \leftrightarrows_n (M, x)$)  е.т.е. у $\exists$лоизы есть выигрышная стратегия в игре $BG_{n}[(M, x), (M', x')]$.
\end{block}
}

\frame{
\begin{exampleblock}{Теорема} \centering
$(M, x) \leftrightarrows_n (M', x') \iff (M, x) \equiv_n (M', x')  $
\end{exampleblock}

\begin{proof}($\Rightarrow$)	
\end{proof}
}


\frame{\frametitle{Две последовательности моделей}
\begin{exampleblock}{Утверждение}
$(M_{n+1}, x) \equiv_n (M'_{n+1},x)$
\begin{proof} В силу $(M_{n+1}, x) \leftrightarrows_n (M'_{n+1},x)$. 
\end{proof}
\end{exampleblock}
}

\frame{\frametitle{$EL\prec_{S5} ELC$}
$EL\preceq ELC$ – очевидно. Следовательно $EL\preceq_{S5} ELC$. Докажем, что $ELC \not\preceq_{S5} EL$. 
}

\frame{
$EL\preceq ELC$ – очевидно. Следовательно $EL\preceq_{S5} ELC$. Докажем, что $ELC \not\preceq_{S5} EL$. 
Допустим, что найдется $\varphi^* \in EL$ т.ч.$\varphi^* \equiv C_{ab} p$. Докажем, что такое допущение приводит к противоречию. Обозначим,  $md (\varphi^*)=n$. 
\begin{prooftree}
\hypo{M_{n+1}, x  \models C_{ab} p}
\hypo{\varphi^* \equiv C_{ab} p}
\infer2{M_{n+1}, x  \models \varphi^*}

\hypo{(M_{n+1}, x) \equiv_n (M'_{n+1},x)}

\infer2{M'_{n+1}, x \models \varphi^*}

\hypo{M'_{n+1}, x  \not \models C_{ab} p}
\hypo{\varphi^* \equiv C_{ab} p}
\infer2{M'_{n+1}, x \not \models \varphi^*}

\infer2 \bot
\end{prooftree}
}



\section{Лаконичность языка}
\frame{\frametitle{Лаконичность языка}
$\EL=\ELE$, но $\ELE$ более лаконичный, более компактный.

Две последовательности формул:
\begin{itemize}
\item $\alpha_n: = \neg E^n_{ab} p$ 
\item $\beta_1:=  \neg (K_a p \wedge K_b p)$
\item $\beta_n:= \neg (K_a \neg \beta_{n-1} \wedge K_b \neg \beta_{n-1} )$

\end{itemize}

\begin{center}
\begin{tabular}{lll}
$n$& $\alpha_n$  & $\beta_n$ \\ 
\hline
1 & $\neg E_{ab}p$ & $\neg (K_a p \wedge K_b p)$ \\
2 & $\neg E^{2}_{ab}p$ & $\neg (K_a p \wedge K_b p \wedge K_a K_b p \wedge K_b K_a p)$ \\
3 & $\neg E^{3}_{ab}p$ & $\neg (K_a p \wedge K_b p \wedge K_a K_b p \wedge K_b K_a p 
\wedge K_b K_a K_b p \wedge K_a K_b K_a p)$ 
\end{tabular}
\end{center}
} 

\frame{ \frametitle{Схема сравнения языков по выразительной силе}
}

\end{document}