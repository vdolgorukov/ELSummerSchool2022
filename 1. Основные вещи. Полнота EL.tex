%----------------------------------------------------------------------------------------
%	PACKAGES AND THEMES
%----------------------------------------------------------------------------------------
\documentclass[aspectratio=169,xcolor=dvipsnames]{beamer}
\usetheme{SimplePlus}

\usepackage[utf8]{inputenc} 
\usepackage[T2A]{fontenc} 
\usepackage[russian]{babel}
\usepackage{amsmath,amssymb}
\usepackage{hyperref}
\usepackage{graphicx} % Allows including images
\usepackage{booktabs} % Allows the use of \toprule, \midrule and \bottomrule in tables
\usepackage{multicol}
\usepackage{expl3,xparse}
\usepackage{ebproof}
\usepackage{mathtools} 
\usepackage{extarrows} 
\usepackage{fitch}

%%%Настройка определений, теорем, лемм и пр
\usepackage{amsthm}
\theoremstyle{remark}
\theoremstyle{mystyle}
\newtheorem{mydef}{Определение}
\newtheorem{mylemma}[theorem]{Лемма}
\newtheorem{mycor}[theorem]{Утверждение}
%%%%%%%%%%%%%%%%%%%%%%%%%%%%%%%%%%%%%%%%%%%%%%%%%

\setbeamersize{text margin left=15pt,text margin right=15pt}

%Команды
%\newcommand{\mydef}[1]{\textbf{Определение. }\alert{#1}}
\newcommand{\mythm}[1]{\textbf{Теорема. }\alert{#1}}
\newcommand{\mylem}[1]{\textbf{Лемма. }\alert{#1}}
\newcommand{\myprop}[1]{\textbf{Утверждение. }\alert{#1}}

%Сокращения для названий языков
\newcommand{\EL}{\mathcal{EL}}
\newcommand{\ELC}{\mathcal{EL\text{-}C}}
\newcommand{\ELD}{\mathcal{EL\text{-}D}}
\newcommand{\ELCD}{\mathcal{EL\text{-}CD}}
\newcommand{\ELE}{\mathcal{EL\text{-}E}}
\newcommand{\PAL}{\mathcal{PAL}}
\newcommand{\PALC}{\mathcal{PAL\text{-}C}}
\newcommand{\PALD}{\mathcal{PAL\text{-}D}}
\newcommand{\PALCD}{\mathcal{PAL\text{-}CD}}
\newcommand{\ELRC}{\mathcal{EL\text{-}RC}}
\newcommand{\PALRC}{\mathcal{PAL\text{-}RC}}

%Еще сокращения
\newcommand{\metabot}{\text{<<}\bot\text{>>}}

%----------------------------------------------------------------------------------------
%	TITLE PAGE
%----------------------------------------------------------------------------------------
\subtitle{Мини-курс <<Эпистемическая логика: исчисления и модели>>}

\author[]{Виталий Долгоруков, Елена Попова}

\institute[] % Your institution as it will appear on the bottom of every slide, may be shorthand to save space
{Международная лаборатория логики, лингвистики и формальной философии НИУ ВШЭ}
\date{Летняя школа <<Логика и формальная философия>> \\ Факультет свободных искусств и наук \\ 
сентябрь 2022} % Date, can be changed to a custom date


%----------------------------------------------------------------------------------------
%	PRESENTATION SLIDES
%----------------------------------------------------------------------------------------
 \title{Исчисления $K_m$, $S4_m$, $S5_m$: корректность и полнота}

\usepackage{expl3, xparse}
\usepackage{ebproof}

\begin{document}  
\frame{\titlepage}

%\frame{\frametitle{Введение}
%\begin{itemize}
%\item Эпистемическая логика описывает знание в мультиагентных системах 
%\item Приложения эпистемической логики: формальная эпистемология, эпистемическая теория игр, когнитивные науки, робототехника и др.  
%\item В этом курсе планируется больше чистой логики, чем формальной эпистемологии
%\item Приветствуется знакомство с базовыми вещами в модальной логике
%\item Темы: статическая эпистемическая логика, групповые расширения, динамическая эпистемическая логика
%\item Порядок: модели, исчисления, полнота и корректность, сравнение языков по выразительности 
%\end{itemize}
%}

\frame{\frametitle{Исчисления}
\begin{block}{Исчисление $K_m$}
Аксиомные схемы:
\begin{itemize}
\item Тавтологии КЛВ
\item $K_i(\varphi \to \psi) \to  (K_i \varphi \to K_i \psi)$	
\end{itemize}
Правила вывода:

\begin{multicols}{2}
\begin{prooftree}
\hypo{\varphi}
\hypo{\varphi \to \psi}
\infer2[MP]{\psi}
\end{prooftree}
	
\begin{prooftree}
\hypo{\varphi}
\infer1[G]{K_i \varphi}
\end{prooftree}
	
\end{multicols}
\end{block}
}


\frame{\frametitle{Расширения $K_m$}
\begin{block}{$S4_{m}$}
$K_m$, к которому добавили следующие аксиомные схемы:
\begin{itemize}
\item $K_i \varphi \to \varphi$
\item $K_i \varphi \to K_i K_i \varphi$	
\end{itemize}
\end{block}

\begin{block}{$S5_{m}$}
$S4_{m}$, к которому добавили:
\begin{itemize}
\item $\varphi \to K_i \hat{K}_i \varphi$	
\end{itemize}
или 
$S5_{m}$ – $K_{m}$, к которому добавили:
\begin{itemize}
\item $K_i \varphi \to \varphi$
\item $\neg K_i \varphi \to K_i \neg K_i  $
\end{itemize}	
\end{block}
}

\frame{\frametitle{Вывод}
\begin{block}{Определение} \textbf{Выводом }($\vdash_{L} \varphi$) в исчислении $L$ будем называть последовательность формул $\varphi_1, \dots, \varphi_n=\varphi$, в которой каждая формула $\varphi_i$ ($1 \leq i \leq n $) является подстановочным случаем аксиомной схемы или получена из предыдущих формул по одному из правил вывода. 
\end{block}
}

\frame{\frametitle{Пример}
\begin{block}{$\vdash  K_i (p \wedge q) \to K_i p$}
1. $(p \wedge q) \to  p$ – тавтология \\
2. $K_i((p \wedge q) \to  p)$ – из 1 по $G$ \\
3. $K_i((p \wedge q) \to  p) \to (K_i (p \wedge q) \to  K_ip)$ – акс. $K$ \\
4. $K_i (p \wedge q) \to  K_ip$ – MP 3,4
\end{block}

\pause

\begin{exampleblock}{Упражнение} Докажите, что
\begin{center}
\begin{prooftree}
\hypo{(\varphi_1 \wedge \dots \wedge \varphi_n) \to \psi}
\infer1{(K_i \varphi_1 \wedge \dots \wedge K_i \varphi_{n}) \to K_i \psi}
\end{prooftree}
\end{center}
\end{exampleblock}

}

\frame{\frametitle{Вывод из гипотез (локальный)}
\begin{block}{Определение} Будем говорить, что \textbf{$\varphi$ (локально) выводится из множества гипотез $\Gamma$} в исчислении $L$ ($\Gamma \vdash_{L} \varphi$), если найдутся  $\varphi_1, \dots, \varphi_n \in \Gamma$ т.ч. $\vdash_{L} (\varphi_1 \wedge \dots \wedge \varphi_n) \to \varphi$
\end{block}

Примеры:
\begin{itemize}
\item $p \not \vdash \Box p$
\item $\Box p, \Box q \vdash \Box (p \wedge q)$

\end{itemize}


}

\frame{\frametitle{Полнота и корректность}
\begin{itemize}
\item $L \in \{ K_m, S5_m, S4_m, \dots \}$ 
\item $C \in \{ Ref, Eq, Pre, \dots  \}$
\item (сильная) корректность \\
$\Gamma \vdash_{L} \varphi \Rightarrow \Gamma \models_{C} \varphi $
\item (сильная) полнота \\
$\Gamma \models_{C} \varphi  \Rightarrow \Gamma \vdash_{L} \varphi $
\end{itemize}

}

\frame{\frametitle{Сборка доказательства}
Корректность: упражнение. 

\begin{center}
$X \not \vdash_{L} \varphi \Rightarrow  
X, \neg \varphi \not   \vdash_{L} \bot \Rightarrow
(X \cup \{\neg \varphi\}) \subseteq X' \Rightarrow
  M^c, X' \models X \text{ и } M^c, X' \models \neg \varphi \Rightarrow$ \\
$(M^c \in C \Rightarrow \; X \not \models_{C} \varphi )$ 
\end{center}
}

\frame{\frametitle{Макимальность и непротиворечивость}
\begin{block}{}\textbf{Определение.} Пусть $L \in \{K, S4, S5, \dots \}$. Множество формул $X$ будем называть \textbf{$L$-непротиворечивым}, если: $X \not \vdash_{L} \bot$. 
\end{block}

\begin{block}{}\textbf{Определение.} Пусть $L \in \{K, S4, S5, \dots \}$.  Множество формул $X$ будем называть \textbf{максимальным $L$-непротиворечивым множеством}, если $X$ – непротиворечиво и для любой формулы $\varphi$: $\varphi \in X$ или  $\neg \varphi \in X$.
\end{block}
}

\frame{\frametitle{Свойства м.L-н.м.}
\begin{block}{}. \textbf{Утверждение.} Пусть $X$ – м.L-н.м., тогда
\begin{enumerate}
	\item $\neg \varphi \in X \Leftrightarrow \varphi \not \in X$
	\item $\varphi \wedge \psi \in X \Leftrightarrow  (\varphi \in X \text{ и } \psi \in X)$ 
	\item $\varphi \vee \psi \in X \Leftrightarrow  (\varphi \in X \text{ или } \psi \in X)$ 
	\item $\varphi \to \psi \in X \Leftrightarrow  (\varphi \in X \Rightarrow \psi \in X)$ 
	\item $\varphi \in X \Leftrightarrow X \vdash_{L} \varphi$
	\item $\bot \not \in X$
\end{enumerate}
	
\end{block}

}


\frame{\frametitle{Каноническая модель}
\begin{block}{}\textbf{Определение}. \textbf{Канонической моделью} будем называть следующую структуру $M^c = (W^c, (R_i^c)_{i \in Ag}, V^c )$, где
\begin{itemize}
\item $W^c:= \{ X \subset \EL \mid X - \text{максимальное L-непротиворечивое множество} \}$ 
\item $XR_i^cY \Leftrightarrow \forall \varphi \in \EL (K_i \varphi \in X \Rightarrow \varphi \in Y)$	
\item $V^c(p):= \{ X \in W^c \mid p \in X \} $, т.е. $M^c, X \models p \Leftrightarrow p \in X$
\end{itemize}
\end{block}
}


\frame{\frametitle{Лемма Линденбаума}
\begin{block}{}\textbf{Лемма}. Пусть $X$ – непротиворечивое множество формул, тогда найдется $X'$ т.ч. $X \in \subseteq X'$ и $X'$ - м.L-н.м. формул.
\end{block}
Занумеруем множество всех формул языка $\EL = \{ \varphi_1, \varphi_2, \dots \}$ Рассмотрим следующее множество: 
\begin{itemize}
\item $X_1 = X$
\item $X_{n+1} = 
\begin{cases}
	X_n \cup \{ \varphi_{n+1}\} , \text{ если } X_n \cup \{ \varphi_{n+1}\} \text{ непротиворечиво} \\
	X_n \cup \{ \neg \varphi_{n+1}\}, \text{ иначе } 
\end{cases}
$	
\item $X' = \bigcup \limits_{i=0}^{\infty} X_i $
\end{itemize}
Допустим, что и $X_n \cup \{ \varphi_{n+1}\}$, и  $X_n \cup \{ \neg \varphi_{n+1}\}$ – противоречивы. Тогда, $X_n \vdash \neg \varphi_{n+1}$ и $X_n \vdash \varphi_{n+1}$. То есть, $X_n$ – противоречиво. 
}

\frame{\frametitle{Лемма об истинности}
\begin{block}{Лемма} Пусть $X \in W^c$, $\varphi' \in \EL$, тогда  
$\varphi' \in X \Leftrightarrow M^c, X \models \varphi'$

\end{block}
Докажем индукцией по построению $\varphi'$.
\begin{description}
\item[База индукции] $\varphi'= p $ \\
$p \in X \Leftrightarrow M^c, X \models p$ (по опр. )

\item[Шаг индукции] Рассмотрим случаи $\varphi' = \neg \varphi$, $\varphi' = \varphi \wedge \psi$,  $\varphi'= K_i \varphi$ \\ 
$\neg \varphi$, $\varphi \wedge \psi$ –  упражнение
\end{description}
}


\frame{ \frametitle{Случай $K_i \varphi$ $\Rightarrow$}
\begin{fitch}
K_i \varphi \in X & $\rhd \; M^c, X \models K_i \varphi \Leftrightarrow  \rhd \; \forall Y (XR_i^cY \Rightarrow M^c, Y \models  \varphi) $ \\
\fh \fw{Y} \; XR_i^cY & $\rhd \; M^c, Y \models  \varphi$ \\
\fa \varphi \in Y  & 1,2 \\
\fa   M^c, Y \models  \varphi & 3 по ПИ \\
\forall Y (XR_i^cY \Rightarrow M^c, Y \models  \varphi) & 2-4 \\
 M^c, X \models K_i \varphi   & 4 по опр.
\end{fitch}
}


\frame{
\frametitle{Случай $K_i \varphi$ $\Leftarrow$}
\begin{fitch}
K_i \varphi \not \in X & $\rhd \; M^c, X \not \models  K_i \varphi \Leftrightarrow \rhd \; \exists Y (XR^c_iY \wedge M^c, Y \not \models  \varphi)$ \\
\fh y_0 = \{\psi \mid K_i \psi \in X \} \cup \{ \neg \varphi \} \vdash \bot   & $\rhd \; \metabot$ \\
\fa \psi_1, \dots, \psi_n, \neg \varphi \vdash \bot & из 2\\
\fa \psi_1, \dots, \psi_n  \vdash \varphi & из 3\\
\fa \vdash (\psi_1 \wedge \dots \wedge \psi_n) \to \varphi  & из 4\\
\fa \vdash (K_i \psi_1 \wedge \dots \wedge K_i \psi_n) \to K_i \varphi & из 5 по упр. \\
\fa X \vdash (K_i \psi_1 \wedge \dots \wedge K_i \psi_n) \to K_i \varphi & из 6 \\
\fa X \vdash (K_i \psi_1 \wedge \dots \wedge K_i \psi_n) & по построению $\psi$ \\
\fa X \vdash K_i \varphi & из 7, 8 \\
\fa \neg K_i \varphi \in X & из 1
\end{fitch}
}

\frame{ \frametitle{Случай $K_i \varphi$ $\Leftarrow$}
%
%\ftag{11}{
%\fa \fa \vdash \underline{X} \to K_i \varphi 
%} \setcounter{fitchcounter}{11} \\

\begin{fitch}
\ftag{11}{
\fa X \vdash \neg K_i } \setcounter{fitchcounter}{11}  из 10 \\
\fa \metabot & 9, 11\\
y_0 \not \vdash \bot & из 2-12\\
y_0 \subseteq Y \in W^c & по л. Линденбаума из 13\\
XR^c_iY & из 2 \\
\neg \varphi \in Y & из 2 \\
\varphi \not \in Y & из 16 \\
M^c, Y \not \models \varphi & из 17 по ПИ \\
\exists Y (XR^c_iY \wedge M^c, Y \not \models  \varphi) & из 15, 18\\
M^c, X \not \models  K_i \varphi & из 19
\end{fitch}
}


\frame{\frametitle{Классы моделей}
\begin{block}{}\textbf{Утверждения.} 
\begin{enumerate}
\item Если $L=KT$, то $R^c$ – рефлексивно
\item Если $L=K4$, то $R^c$ – транзитивно
\item Если $L=KB$, то $R^c$ – симметрично	
\end{enumerate}

	
\end{block}

}



\end{document}
