%----------------------------------------------------------------------------------------
%	PACKAGES AND THEMES
%----------------------------------------------------------------------------------------
\documentclass[aspectratio=169,xcolor=dvipsnames]{beamer}
\usetheme{SimplePlus}

\usepackage[utf8]{inputenc} 
\usepackage[T2A]{fontenc} 
\usepackage[russian]{babel}
\usepackage{amsmath,amssymb}
\usepackage{hyperref}
\usepackage{graphicx} % Allows including images
\usepackage{booktabs} % Allows the use of \toprule, \midrule and \bottomrule in tables
\usepackage{multicol}
\usepackage{expl3,xparse}
\usepackage{ebproof}
\usepackage{mathtools} 
\usepackage{extarrows} 
\usepackage{fitch}
\theoremstyle{plain}
\newtheorem{mydef}{Определение}
\setbeamersize{text margin left=15pt,text margin right=15pt}

%Сокращения для названий языков
\newcommand{\EL}{\mathcal{EL}}
\newcommand{\ELC}{\mathcal{EL\text{-}C}}
\newcommand{\ELD}{\mathcal{EL\text{-}D}}
\newcommand{\ELCD}{\mathcal{EL\text{-}CD}}
\newcommand{\ELE}{\mathcal{EL\text{-}E}}
\newcommand{\PAL}{\mathcal{PAL}}
\newcommand{\PALC}{\mathcal{PAL\text{-}C}}
\newcommand{\PALD}{\mathcal{PAL\text{-}D}}
\newcommand{\PALCD}{\mathcal{PAL\text{-}CD}}
\newcommand{\ELRC}{\mathcal{EL\text{-}RC}}
\newcommand{\PALRC}{\mathcal{PAL\text{-}RC}}

%----------------------------------------------------------------------------------------
%	TITLE PAGE
%----------------------------------------------------------------------------------------
\subtitle{Мини-курс <<Эпистемическая логика: исчисления и модели>>}

\author[]{Виталий Долгоруков, Елена Попова}

\institute[] % Your institution as it will appear on the bottom of every slide, may be shorthand to save space
{Международная лаборатория логики, лингвистики и формальной философии НИУ ВШЭ}
\date{Летняя школа <<Логика и формальная философия>> \\ Факультет свободных искусств и наук \\ 
сентябрь 2022} % Date, can be changed to a custom date


%----------------------------------------------------------------------------------------
%	PRESENTATION SLIDES
%----------------------------------------------------------------------------------------
 \title{Основные понятия}

\begin{document}  





%%TODO разобраться c \neg \Gamma и умным отрицанием
%\frame{\frametitle{Замыкание}
%\begin{block}{Обозначение}
%$\Gamma \subseteq L_{CK}$, определим
%	$\neg \Gamma := \{\neg \varphi \mid \varphi \in \Gamma \text{ и } \varphi \text{ не начинается с } "\neg" \}$.
%Пример: пусть $\Gamma = \{p, \neg K_i q, r, \neg s\}$, тогда $\neg \Gamma = \{\neg p, \neg r\}$	
%\end{block}
%
%%TODO Замыкание Фишера-Ладнера
%\begin{block}{} Определение. \alert{Замыкание $cl(\varphi)$}.
%Для $\varphi \in L_{KC}$ определим четыре множества:
%$cl_1(\varphi) \subset cl_2(\varphi) \subset cl_3 (\varphi) \subset cl(\varphi)$.
%\begin{description}
%\item[$cl_1(\varphi)$] – наименьшее множество, замкнутое по следующим правилам:
%\begin{enumerate}
%\item $\varphi \in cl_1(\varphi)$
%\item если $\psi \in cl_1(\varphi)$, то $Sub(\psi) \subseteq cl_1(\varphi)$
%\item если $C_G \psi \in cl_1(\varphi)$, то $\{K_i C_G \psi \mid i \in G \} \subseteq cl_1(\varphi)$	
%\end{enumerate}
%\item[$cl_2(\varphi)$]$:= cl_1(\varphi) \cup \{ \neg \psi \mid \psi \in cl_1(\varphi) \text{ и } \psi \not = \neg \dots \}$	 
%\item[$cl_3(\varphi)$]$:= cl_2(\varphi) \cup \{K_i K_i \psi \mid K_i \psi \in cl_2(\varphi) \} \cup \{K_i \neg K_i \psi \mid \neg K_i \psi \in cl_2(\varphi) \}$	
%\item[$cl(\varphi)$]$:=cl_3(\varphi) \cup \{ \neg \psi \mid \psi \in cl_3(\varphi) \text{ и }  \psi \not = \neg \dots \}$
%\end{description}
%\end{block}
%}

%TODO добавить q и ~q!
%\frame{
%\begin{exampleblock}{Пример $cl_1(\varphi) \subset cl_2(\varphi) \subset cl_3 (\varphi) \subset cl(\varphi)$}
%\centering 
%\scalebox{0.75}{
%\begin{tabular}{|c|c|c|c|c|}
%\hline
%$\varphi$               & $cl_1(\varphi)$         & $cl_2(\varphi)$                 & $cl_3(\varphi)$                 & $cl(\varphi)$                   \\ \hline
%$\neg p \wedge C_{ab}q$ & $\neg p \wedge C_{ab}q$ & $\neg p \wedge C_{ab}q$         & $\neg p \wedge C_{ab}q$         & $ \neg p \wedge C_{ab}q$        \\ \hline
%                        &                         & $\neg (\neg p \wedge C_{ab} q)$ & $\neg (\neg p \wedge C_{ab} q)$ & $\neg (\neg p \wedge C_{ab} q)$ \\ \hline
%                        & $\neg p$                & $\neg p$                        & $\neg p$                        & $\neg p$                        \\ \hline
%                        & $p$                     & $p$                             & $p$                             & $p$                             \\ \hline
%                        & $C_{ab}q$               & $C_{ab}q$                       & $C_{ab}q$                       & $C_{ab}q$                       \\ \hline
%                        &                         & $\neg C_{ab}q$                  & $\neg C_{ab}q$                  & $\neg C_{ab}q$                  \\ \hline
%                        & $q$                     & $q$                             & $q$                             & $q$                             \\ \hline
%                        &                         & $\neg q$                        & $\neg q$                        & $\neg q$                        \\ \hline
%                        & $K_a C_{ab}q$           & $K_a C_{ab}q$                   & $K_a C_{ab}q$                   & $K_a C_{ab}q$                   \\ \hline
%                        &                         & $\neg K_a C_{ab}q$              & $\neg K_a C_{ab}q$              & $\neg K_a C_{ab}q$              \\ \hline
%                        & $K_bC_{ab}q$            & $K_bC_{ab}q$                    & $K_bC_{ab}q$                    & $K_bC_{ab}q$                    \\ \hline
%                        &                         & $\neg K_bC_{ab}q$               & $\neg K_bC_{ab}q$               & $\neg K_bC_{ab}q$               \\ \hline
%                        &                         &                                 & $K_a K_a C_{ab}q$               & $K_a K_a C_{ab}q$               \\ \hline
%                        &                         &                                 &                                 & $\neg K_a K_a C_{ab}q$          \\ \hline
%                        &                         &                                 & $K_a \neg K_a C_{ab}q$          & $K_a \neg K_a C_{ab}q$          \\ \hline
%                        &                         &                                 &                                 & $\neg K_a \neg K_a C_{ab}q$     \\ \hline
%                        &                         &                                 & $K_b K_b C_{ab}q$               & $K_b K_b C_{ab}q$               \\ \hline
%                        &                         &                                 &                                 & $\neg K_b  K_b C_{ab}q$         \\ \hline
%                        &                         &                                 & $K_b \neg K_b C_{ab}q$          & $K_b \neg K_b C_{ab}q$          \\ \hline
%                        &                         &                                 &                                 & $K_b \neg K_b C_{ab}q$          \\ \hline
%\end{tabular}
%}	
%\end{exampleblock}
%}

\end{document}